\section{Categorical data models}
\label{subsec:categoricalDataModels}


%%%%%%%%%%%%%%%%%%%%%%%%%%%%%%%%%%%%%%%%%%%%%%%%%%%%%%%%%%%%%%%%
\subsection{Binomial distribution}

\paragraph{Observation model}
	
\begin{itemize}
\item
Type of observed variable -- discrete/categorical
\item
Category variable: $y$
\item
Set of categories: $\{0,1\}$
\item
Probability for category '1'
\begin{eqnarray}
&& P(y=1) = p \nonumber
\end{eqnarray}
\end{itemize}


\subsubsection{NM-TRAN code:}
\myStartLine
\lstset{language=NONMEMdataSet}
\begin{lstlisting}
$PROB Binary model

$INPUT ID TIME DV

$DATA data.csv IGNORE=@ 

$PRED
  TVBASE = THETA(1)                      ; TVBASE is a proportion
  PHI    = LOG(TVBASE/(1-TVBASE)) + ETA(1)  
  BASE   = EXP(PHI)/(1+EXP(PHI))   ; Logit transformation, 0<BASE<1

  IF(DV.GT.1) THEN   ; A event for binary data
    Y=BASE
    RDV = 1     ; the "real" DV
  ENDIF
  IF(DV.LE.1) THEN   ; No event for binary data
    Y=1-BASE
    RDV = 0     ; the "real" DV
  ENDIF

$THETA (0,.8)    ; BASE
$OMEGA 0.1       ; BSV BASE

$ESTIM MAXEVAL=9999 METHOD=COND LAPLACE LIKE PRINT=1 MSFO=msfb
\end{lstlisting}
\myEndLine

\subsubsection{MLXTRAN code:}
Code source: Lixoft MLXTRAN tutorial.

\myStartLine
\lstset{language=MLXTRANcode}
\begin{lstlisting}
DESCRIPTION:
Categorical data model - Binomial distribution

INPUT:
parameter = p

OBSERVATION:
y= { 
	type = categorical 
	categories = {0,1} 
	P(y=1) = p 
}

OUTPUT:
output = y
\end{lstlisting}
\myEndLine

\subsubsection{PharmML code:}

\lstset{language=XML}
\begin{lstlisting}
        <ObservationModel blkId="om1">
            <Discrete>
                <CategoricalData ordered="no">
                    <SimpleParameter symbId="p"/>
                    
                    <ListOfCategories> 
                        <Category symbId="cat0"/>
                        <Category symbId="cat1"/>
                    </ListOfCategories>
                    
                    <CategoryVariable symbId="y"/>
                    
                    <!-- P(y = 1) = p -->
                    <ProbabilityAssignment>
                        <Probability linkFunction="identity">
                            <math:LogicBinop op="eq">
                                <ct:SymbRef symbIdRef="y"/>
                                <ct:SymbRef symbIdRef="cat1"/>
                            </math:LogicBinop>
                        </Probability>
                        <ct:Assign>
                            <ct:SymbRef symbIdRef="p"/>
                        </ct:Assign>
                    </ProbabilityAssignment>
                </CategoricalData>
            </Discrete>
        </ObservationModel>
\end{lstlisting}


\subsubsection{PharmML code -- alternative using UncertML:}

\lstset{language=XML}
\begin{lstlisting}
        <ObservationModel blkId="om1">
            <Discrete>
                <CategoricalData ordered="no">
                    <SimpleParameter symbId="p"/>
                    
                    <ListOfCategories> 
                        <Category symbId="cat0"/>
                        <Category symbId="cat1"/>
                    </ListOfCategories>
                    
                    <CategoryVariable symbId="y"/>
                    
                    <PMF linkFunction="identity">
                        <BernoulliDistribution xmlns="http://www.uncertml.org/3.0" 
                            definition="http://www.uncertml.org/3.0">
                            <categoryProb definition="http://www.uncertml.org/3.0">
                                <name>cat1</name>
                                <probability>
                                    <var varId="p"/>
                                </probability>
                            </categoryProb>
                        </BernoulliDistribution>
                    </PMF>
                </CategoricalData>
            </Discrete>
        </ObservationModel>
\end{lstlisting}


%%%%%%%%%%%%%%%%%%%%%%%%%%%%%%%%%%%%%%%%%%%%%%%%%%%%%%%%%%%%%%%%
\subsection{Binomial distribution with Markovian dependence}

\paragraph{Observation model}

% TEMPLATE for CATEGORICAL MODEL DESCRIPTION
\begin{itemize}
\item
Type of observed variable -- discrete/categorical
\item
Category variable: $y$
\item
Previous state variable: $yp$
\item
Set of categories: $\{0,1\}$
\item
Probability for category '1'
\begin{align}
& P(y=1) = p \nonumber
\end{align}
\item
Markov dependance with transition probabilities:
\begin{itemize}
\item
$P(y=1 | yp=0) = p01$ and 
\item
$P(y=1 | yp=1) = p11$
\end{itemize}
\end{itemize}

\begin{figure}[htbp!]
\includegraphics[width=.75\textwidth]{pics/binomialMarkov} 
\caption{Markov transition probabilities for two-states model.}
\label{fig:binomialMarkov}
\end{figure}


\subsubsection{NM-TRAN code:}

\myStartLine
\lstset{language=NONMEMdataSet}
\begin{lstlisting}
$PROB Transition probabilities

$INPUT ID TIME DV PDV ;PDV=value of preceding observation

$DATA data.csv IGNORE=@ 

$PRED

  TVP10 = THETA(1)                    
  PHI1    = LOG(TVP10/(1-TVP10)) + ETA(1)  
  P10   = EXP(PHI1)/(1+EXP(PHI1))
  
  TVP01 = THETA(2)                    
  PHI0    = LOG(TVP01/(1-TVP01)) + ETA(2)  
  P01   = EXP(PHI0)/(1+EXP(PHI0))
   
   IF(PDV.EQ.0.AND.DV.EQ.1) Y=P10
   IF(PDV.EQ.0.AND.DV.EQ.0) Y=1-P10 
   IF(PDV.EQ.1.AND.DV.EQ.0) Y=P01 
   IF(PDV.EQ.1.AND.DV.EQ.1) Y=1-P01 
   
$THETA  (0,.1,1)   
$THETA  (0,.1,1)   
$OMEGA 0.1 
$OMEGA 0.1  

$ESTIM MAXEVAL=9999 METHOD=COND LAPLACE LIKE PRINT=1 MSFO=msfb
\end{lstlisting}
\myEndLine

\subsubsection{MLXTRAN code:}
Code provided by Marc Lavielle. 

\myStartLine
\lstset{language=MLXTRANcode}
\begin{lstlisting}
DESCRIPTION:
Categorical data model with Markovian dependence, Binomial distribution

INPUT:
parameter = {p01, p11}

OBSERVATION:
y = {
      type = categorical
      categories = {0,1}
      dependence = Markov
      P(y=1 | yp=0) = p01
      P(y=1 | yp=1) = p11
}

OUTPUT:
output = y
\end{lstlisting}
\myEndLine

\subsubsection{PharmML code:}

\lstset{language=XML}
\begin{lstlisting}
        <ObservationModel blkId="om1">
            <Discrete>
                <CategoricalData ordered="no">
                    <SimpleParameter symbId="p01"/>
                    <SimpleParameter symbId="p11"/>
                    
                    <ListOfCategories> 
                        <Category symbId="cat0"/>
                        <Category symbId="cat1"/>
                    </ListOfCategories>
                    
                    <CategoryVariable symbId="y"/>
                    <PreviousStateVariable symbId="yp"/>
                    
                    <Dependance type="discreteMarkov"/>
                    
                    <!-- P(y=1|yp=0)=p01 -->
                    <ProbabilityAssignment>
                        <Probability>
                            <CurrentState>
                                <math:LogicBinop op="eq">
                                    <ct:SymbRef symbIdRef="y"/>
                                    <ct:SymbRef symbIdRef="cat1"/>
                                </math:LogicBinop>
                            </CurrentState>
                            <PreviousState>
                                <math:LogicBinop op="eq">
                                    <ct:SymbRef symbIdRef="yp"/>
                                    <ct:SymbRef symbIdRef="cat0"/>
                                </math:LogicBinop>
                            </PreviousState>
                        </Probability>
                        <ct:Assign>
                            <ct:SymbRef symbIdRef="p01"/>
                        </ct:Assign>
                    </ProbabilityAssignment>
                    
                    <!-- P(y=1|yp=1)=p11-->
                    <ProbabilityAssignment>
                        <Probability>
                            <CurrentState>
                                <math:LogicBinop op="eq">
                                    <ct:SymbRef symbIdRef="y"/>
                                    <ct:SymbRef symbIdRef="cat1"/>
                                </math:LogicBinop>
                            </CurrentState>
                            <PreviousState>
                                <math:LogicBinop op="eq">
                                    <ct:SymbRef symbIdRef="yp"/>
                                    <ct:SymbRef symbIdRef="cat1"/>
                                </math:LogicBinop>
                            </PreviousState>
                        </Probability>
                        <ct:Assign>
                            <ct:SymbRef symbIdRef="p11"/>
                        </ct:Assign>
                    </ProbabilityAssignment>
                </CategoricalData>
            </Discrete>
        </ObservationModel>
\end{lstlisting}


%%%%%%%%%%%%%%%%%%%%%%%%%%%%%%%%%%%%%%%%%%%%%%%%%%%%%%%%%%%%%%%%
\subsection{Explicit probabilities per category -- three categories}

\paragraph{Observation model}

\begin{itemize}
\item
Type of observed variable -- discrete/categorical
\item
Category variable: $y$
\item
Set of categories: $\{1,2,3\}$
\item
Probabilities for category '1' and '2'
\begin{eqnarray}
&& P(y=1) = a1/(a1+a2+a3)  \nonumber \\
&& P(y=2) = a2/(a1+a2+a3)  \nonumber 
\end{eqnarray}
\end{itemize}


\subsubsection{NM-TRAN code:}

\myStartLine
\lstset{language=NONMEMdataSet}
\begin{lstlisting}
	missing
\end{lstlisting}
\myEndLine

\subsubsection{MLXTRAN code:}
Code source: Lixoft MLXTRAN tutorial.

\myStartLine
\lstset{language=MLXTRANcode}
\begin{lstlisting}
DESCRIPTION:
Categorical data model - 3 categories

INPUT:
parameter = {a1, a2, a3}

OBSERVATION:
y = {
	type = categorical 
	categories = {1, 2, 3} 
	P(y=1) = a1/(a1+a2+a3) 
	P(y=2) = a2/(a1+a2+a3) 
}

OUTPUT:
output = y
\end{lstlisting}
\myEndLine

\textit{ALTERNATIVELY} (for $P(y=1)$ and $P(y=3)$):

\begin{lstlisting}
OBSERVATION:
y = { 
	type = categorical 
	categories = {1, 2, 3} 
	P(y=1) = a1/(a1+a2+a3) 
	P(y=3) = a3/(a1+a2+a3) 
}
\end{lstlisting}

\subsubsection{PharmML code:}

\lstset{language=XML}
\begin{lstlisting}
        <ObservationModel blkId="om1">
            <Discrete>
                <CategoricalData ordered="no">
                    <SimpleParameter symbId="a1"/>
                    <!-- a1, a2 omitted -->
                    
                    <ListOfCategories> 
                        <Category symbId="cat1"/>
                        <Category symbId="cat2"/>
                        <Category symbId="cat3"/>
                    </ListOfCategories>
                    
                    <CategoryVariable symbId="y"/>
                    
                    <!-- P(y = 1) = a1/(a1+a2+a3) -->
                    <ProbabilityAssignment>
                        <Probability linkFunction="identity">
                            <math:LogicBinop op="eq">
                                <ct:SymbRef symbIdRef="y"/>
                                <ct:SymbRef symbIdRef="cat1"/>
                            </math:LogicBinop>
                        </Probability>
                        <ct:Assign>
                            <math:Equation>
                                <math:Binop op="divide">
                                    <ct:SymbRef symbIdRef="a1"/>
                                    <math:Binop op="plus">
                                        <ct:SymbRef symbIdRef="a1"/>
                                        <math:Binop op="plus">
                                            <ct:SymbRef symbIdRef="a2"/>
                                            <ct:SymbRef symbIdRef="a3"/>
                                        </math:Binop>
                                    </math:Binop>
                                </math:Binop>
                            </math:Equation>
                        </ct:Assign>
                    </ProbabilityAssignment>
                    
                    <!-- P(y = 2) = a2/(a1+a2+a3) -->
                    <ProbabilityAssignment>
                        <Probability linkFunction="identity"> 
                            <math:LogicBinop op="eq">
                                <ct:SymbRef symbIdRef="y"/>
                                <ct:SymbRef symbIdRef="cat2"/>
                            </math:LogicBinop>
                        </Probability>
                        <ct:Assign>
                            <math:Equation>
                                <math:Binop op="divide">
                                    <ct:SymbRef symbIdRef="a2"/>
                                    <math:Binop op="plus">
                                        <ct:SymbRef symbIdRef="a1"/>
                                        <math:Binop op="plus">
                                            <ct:SymbRef symbIdRef="a2"/>
                                            <ct:SymbRef symbIdRef="a3"/>
                                        </math:Binop>
                                    </math:Binop>
                                </math:Binop>
                            </math:Equation>
                        </ct:Assign>
                    </ProbabilityAssignment>
                </CategoricalData>
            </Discrete>
        </ObservationModel>
\end{lstlisting}


%%%%%%%%%%%%%%%%%%%%%%%%%%%%%%%%%%%%%%%%%%%%%%%%%%%%%%%%%%%%%%%%
\subsection{Cumulative probabilities -- three categories}

\paragraph{Observation model}

\begin{itemize}
\item
Type of observed variable -- discrete/categorical
\item
Category variable: $y$
\item
Set of categories: $\{1,2,3\}$
\item
Cumulative probabilities
\begin{eqnarray}
&& P(y<=1) = a1/(a1+a2+a3) \nonumber \\
&& P(y<=2) = (a1+a2)/(a1+a2a31)  \nonumber 
\end{eqnarray}
\end{itemize}


\subsubsection{NM-TRAN code:}

\myStartLine

\lstset{language=NONMEMdataSet}
\begin{lstlisting}
	missing
\end{lstlisting}

\myEndLine

\subsubsection{MLXTRAN code:}
Code source: Monolix 4.1 User Manual.

\myStartLine

\lstset{language=MLXTRANcode}
\begin{lstlisting}
DESCRIPTION:
Ordered categorical data model - 3 categories Cumulative probabilities

INPUT:
parameter = {a1, a2, a3}

OBSERVATION:
y = { 
	type = categorical 
	categories = {1, 2, 3} 
	P(y<=1) = a1/(a1+a2+a3) 
	P(y<=2) = (a1+a2)/(a1+a2a31) 
}

OUTPUT:
output = y
\end{lstlisting}

\myEndLine

\subsubsection{PharmML code:}

\lstset{language=XML}
\begin{lstlisting}
        <ObservationModel blkId="om1">
            <Discrete>
                <CategoricalData ordered="yes">
                    <SimpleParameter symbId="a1"/>
                    <SimpleParameter symbId="a2"/>
                    <SimpleParameter symbId="a3"/>
                    
                    <ListOfCategories> 
                        <Category symbId="cat1"/>
                        <Category symbId="cat2"/>
                        <Category symbId="cat3"/>
                    </ListOfCategories>
                    
                    <CategoryVariable symbId="y"/>
                    
                    <!-- P(y <= 1) = a1/(a1+a2+a3) --> 
                    <ProbabilityAssignment>
                        <Probability linkFunction="identity">
                            <math:LogicBinop op="leq">
                                <ct:SymbRef symbIdRef="y"/>
                                <ct:SymbRef symbIdRef="cat1"/>
                            </math:LogicBinop>
                        </Probability>
                        <ct:Assign>
                            <math:Equation>
                                <math:Binop op="divide">
                                    <ct:SymbRef symbIdRef="a1"/>
                                    <math:Binop op="plus">
                                        <ct:SymbRef symbIdRef="a1"/>
                                        <math:Binop op="plus">
                                            <ct:SymbRef symbIdRef="a2"/>
                                            <ct:SymbRef symbIdRef="a3"/>
                                        </math:Binop>
                                    </math:Binop>
                                </math:Binop>
                            </math:Equation>
                        </ct:Assign>
                    </ProbabilityAssignment>
                    
                    <!-- P(y <= 2) = a2/(a1+a2+a3) -->
                    <ProbabilityAssignment>
                        <Probability linkFunction="identity">
                            <math:LogicBinop op="leq">
                                <ct:SymbRef symbIdRef="y"/>
                                <ct:SymbRef symbIdRef="cat2"/>
                            </math:LogicBinop>
                        </Probability>
                        <ct:Assign>
                            <math:Equation>
                                <math:Binop op="divide">
                                    <ct:SymbRef symbIdRef="a2"/>
                                    <math:Binop op="plus">
                                        <ct:SymbRef symbIdRef="a1"/>
                                        <math:Binop op="plus">
                                            <ct:SymbRef symbIdRef="a2"/>
                                            <ct:SymbRef symbIdRef="a3"/>
                                        </math:Binop>
                                    </math:Binop>
                                </math:Binop>
                            </math:Equation>
                        </ct:Assign>
                    </ProbabilityAssignment>
                </CategoricalData>
            </Discrete>
        </ObservationModel>
\end{lstlisting}


%\subsubsection{Tail probabilities - three categories}
%\paragraph{NM-TRAN code provided by ...}
%%\begin{lstlisting}
%%\end{lstlisting}
%\paragraph{MLXTRAN code from Monolix 4.1 User Manual}
%\begin{lstlisting}
%DESCRIPTION:
%Ordered categorical data model - 3 categories Tail probabilities
%
%INPUT:
%parameter = {a1, a2, a3}
%
%OBSERVATION:
%y= { type = categorical 
%categories = {1, 2, 3} 
%P(y>1) = (a2+a3)/(a1+a2+a3) 
%P(y>2) = a3/(a1+a2+a3) }
%
%OUTPUT:
%output = y
%\end{lstlisting}


%%%%%%%%%%%%%%%%%%%%%%%%%%%%%%%%%%%%%%%%%%%%%%%%%%%%%%%%%%%%%%%%
\subsection{Cumulative logit probabilities -- three categories}

\paragraph{Observation model}

\begin{itemize}
\item
Type of observed variable -- discrete/categorical
\item
Category variable: $y$
\item
Set of categories: $\{1,2,3\}$
\item
Cumulative logits
\begin{align}
& \text{logit}(P(y<=1) )= \theta1  \nonumber \\
& \text{logit}(P(y<=2) )= \theta1+ \theta2  \nonumber 
\end{align}
\end{itemize}


\subsubsection{NM-TRAN code:}

\myStartLine

\lstset{language=NONMEMdataSet}
\begin{lstlisting}
	missing
\end{lstlisting}

\myEndLine

\subsubsection{MLXTRAN code:}
Code source: Monolix 4.1 User Manual

\myStartLine

\lstset{language=MLXTRANcode}
\begin{lstlisting}
DESCRIPTION:
Ordered categorical data model - 3 categories Logit - probabilities

INPUT:
parameter = {theta1, theta2}

OBSERVATION:
y = {
	type = categorical 
	categories = {1, 2, 3} 
	logit(P(y<=1))= theta1 
	logit(P(y<=2))= theta1+theta2
}
OUTPUT:
output = y


ALTERNATIVELY with tails

OBSERVATION:
y = { 
	type = categorical 
	categories = {1, 2, 3} 
	logit(P(t>1))= -theta1 
	logit(P(t>2))= -theta1-theta2
}
\end{lstlisting}

\myEndLine

\subsubsection{PharmML code:}

\lstset{language=XML}
\begin{lstlisting}
        <ObservationModel blkId="om1">
            <Discrete>
                <CategoricalData ordered="yes">
                    <SimpleParameter symbId="theta1"/>
                    <SimpleParameter symbId="theta2"/>
                    
                    <ListOfCategories> 
                        <Category symbId="cat1"/>
                        <Category symbId="cat2"/>
                    </ListOfCategories>
                    
                    <CategoryVariable symbId="y"/>
                    
                    <!-- logit(P(y <= 1)) = theta1 -->
                    <ProbabilityAssignment>
                        <Probability linkFunction="logit">
                            <math:LogicBinop op="leq">
                                <ct:SymbRef symbIdRef="y"/>
                                <ct:SymbRef symbIdRef="cat1"/>
                            </math:LogicBinop>
                        </Probability>
                        <ct:Assign>
                            <ct:SymbRef symbIdRef="theta1"/>
                        </ct:Assign>
                    </ProbabilityAssignment>
                    
                    <!-- logit(P(y <= 2)) = theta1 + theta2 -->
                    <ProbabilityAssignment>
                        <Probability linkFunction="logit">
                            <math:LogicBinop op="leq">
                                <ct:SymbRef symbIdRef="y"/>
                                <ct:SymbRef symbIdRef="cat2"/>
                            </math:LogicBinop>
                        </Probability>
                        <ct:Assign>
                            <math:Equation>
                                <math:Binop op="plus">
                                    <ct:SymbRef symbIdRef="theta1"/>
                                    <ct:SymbRef symbIdRef="theta2"/>
                                </math:Binop>
                            </math:Equation>
                        </ct:Assign>
                    </ProbabilityAssignment>
                </CategoricalData>
            </Discrete>
        </ObservationModel>
 \end{lstlisting}



%%%%%%%%%%%%%%%%%%%%%%%%%%%%%%%%%%%%%%%%%%%%%%%%%%%%%%%%%%%%%%%%
\subsection{Tail logit probabilities}

\paragraph{Observation model}

\begin{itemize}
\item
Type of observed variable -- discrete / ordered categorical
\item
Category variable: $y$
\item
Set of categories: $\{0,1,2,3\}$
\item
Tail logits
\begin{align}
& \text{logit}(P(y \ge 0)) = B1 + \eta1 \nonumber \\
& \text{logit}(P(y \ge 1)) = B1 + B2 + \eta1 \nonumber \\
& \text{logit}(P(y \ge 2)) = B1 + B2 + B3 + \eta1 \nonumber 
\end{align}
with $\eta1 \sim N(0, \omega1)$.
\end{itemize}


\subsubsection{NM-TRAN code:}
Code example provided by Mats Karlsson.

\myStartLine

\lstset{language=NONMEMdataSet}
\begin{lstlisting}
$PROB mixed-effect Ordered Categorical, baseline model
$INPUT ID TIME ODV DOSE ICL IV IKA 
       TYPE SMAX DV=SMXH THR CAV CAVH CON
       ;CNT CNT2 CNT3 HC HC2 HC3 
       ;ETA1 ETA2 ETA3 ETA4

$DATA data.csv IGNORE=@ ACCEPT=(THR.GT.0)

$PRED
  ;Baseline values
  B1 = THETA(1)
  B2 = THETA(2)
  B3 = THETA(3)

  ;Logits for Y>=1, Y>=2, Y>=3
  LGE1 = B1+ETA(1)
  LGE2 = B1+B2+ETA(1)
  LGE3 = B1+B2+B3+ETA(1)

  ;Probabilities for Y>=2, Y>=3
  PGE1 = EXP(LGE1)/(1+EXP(LGE1))
  PGE2 = EXP(LGE2)/(1+EXP(LGE2))
  PGE3 = EXP(LGE3)/(1+EXP(LGE3))

  ;Probabilities for Y=0, Y=1, Y=2, Y=3
  P0 = (1-PGE1) 
  P1 = (PGE1-PGE2)
  P2 = (PGE2-PGE3)
  P3 = PGE3   
 
  ;Select appropriate P(Y=m)
  IF(DV.EQ.0) Y=P0
  IF(DV.EQ.1) Y=P1
  IF(DV.EQ.2) Y=P2
  IF(DV.EQ.3) Y=P3

$THETA (0.63)          ; B1
$THETA (-INF,-.1,0)    ; B2
$THETA (-INF,-.1,0)    ; B3
$OMEGA .1


$ESTIM MAXEVAL=9990 METHOD=COND LAPLACE LIKE PRINT=1 MSFO=msfb45
$COV PRINT=E

$TABLE ID TIME NOPRINT ONEHEADER FILE=sdtab45
$TABLE ID CAV CAVH CON NOPRINT ONEHEADER FILE=cotab45
$TABLE ID DOSE NOPRINT ONEHEADER FILE=catab45
$TABLE ID ICL IV IKA NOPRINT ONEHEADER FILE=patab45
\end{lstlisting}

\myEndLine

\subsubsection{MLXTRAN code:}

\myStartLine

\lstset{language=MLXTRANcode}
\begin{lstlisting}
DESCRIPTION: Ordered categorical model with tail probabilities

[INDIVIDUAL]
INPUT:
parameter = {B1_POP ,B1_OMEGA}

DEFINITION:
B1 =  {distribution=logNormal, mean= B1_POP, sd=B1_OMEGA}


[OBSERVATION]
INPUT:
parameter = {B1, B2, B3}

DEFINITION:
y = {
	type = categorical
	categories = {0, 1, 2, 3}
	logit(P(y>=0)) = B1
	logit(P(y>=1)) = B1 + B2
	logit(P(y>=2)) = B1 + B2 + B3
}

OUTPUT:
output = level
\end{lstlisting}

\myEndLine

\subsubsection{PharmML code:}
The code handles only the encoding of probabilities. The parameters are
defined in \xelem{ParameterModel} \emph{pm1}. $b1$ is assumed to be log-normally distributed with
$\log(b1) = \log(b1_{\rm pop}) + \eta_{b1}$, $b2$ and $b3$ are fixed.

\lstset{language=XML}
\begin{lstlisting}
        <ObservationModel blkId="om1">
            <Discrete>
                <CategoricalData ordered="yes">           
                    <!-- b1 -->
                    <!-- omitted definition with IIV -->
                    <!-- b2 -->
                    <SimpleParameter symbId="b2"/>
                    <!-- b3 -->
                    <SimpleParameter symbId="b3"/>
                    
                    <ListOfCategories> 
                        <Category symbId="cat1"/>
                        <Category symbId="cat2"/>
                        <Category symbId="cat3"/>
                    </ListOfCategories>
                    
                    <CategoryVariable symbId="y"/>
                    
                    <!-- logit(P(y >= 1)) = b1 -->
                    <ProbabilityAssignment>
                        <Probability linkFunction="logit">
                            <math:LogicBinop op="geq">
                                <ct:SymbRef symbIdRef="y"/>
                                <ct:SymbRef symbIdRef="cat1"/>
                            </math:LogicBinop>
                        </Probability>
                        <ct:Assign>
                            <ct:SymbRef symbIdRef="b1"/>
                        </ct:Assign>
                    </ProbabilityAssignment>
                    
                    <!-- logit(P(y >= 2)) = b1 + b2 --> 
                    <ProbabilityAssignment>
                        <Probability linkFunction="logit">
                            <math:LogicBinop op="geq">
                                <ct:SymbRef symbIdRef="y"/>
                                <ct:SymbRef symbIdRef="cat2"/>
                            </math:LogicBinop>
                        </Probability>
                        <ct:Assign>
                            <math:Equation>
                                <math:Binop op="plus">
                                    <ct:SymbRef symbIdRef="b1"/>
                                    <ct:SymbRef symbIdRef="b2"/>
                                </math:Binop>
                            </math:Equation>
                        </ct:Assign>
                    </ProbabilityAssignment>
                    
                    <!-- logit(P(y >= 3)) = b1 + b2 + b3 --> 
                    <ProbabilityAssignment>
                        <Probability linkFunction="logit">
                            <math:LogicBinop op="geq">
                                <ct:SymbRef symbIdRef="y"/>
                                <ct:SymbRef symbIdRef="cat3"/>
                            </math:LogicBinop>
                        </Probability>
                        <ct:Assign>
                            <math:Equation>
                                <math:Binop op="plus">
                                    <math:Binop op="plus">
                                        <ct:SymbRef symbIdRef="b1"/>
                                        <ct:SymbRef symbIdRef="b2"/>
                                    </math:Binop>
                                    <ct:SymbRef symbIdRef="b3"/>
                                </math:Binop>
                            </math:Equation>
                        </ct:Assign>
                    </ProbabilityAssignment>
                </CategoricalData>
            </Discrete>
        </ObservationModel>
\end{lstlisting}


%%%%%%%%%%%%%%%%%%%%%%%%%%%%%%%%%%%%%%%%%%%%%%%%%%%%%%%%%%%%%%%%
\subsection{Cumulative logit probabilities with Markovian dependence}

\paragraph{Observation model}

\begin{itemize}
\item
Type of observed variable -- discrete / ordered categorical
\item
Category variable: $y$
\item
Previous state variable: $yp$
\item
Set of categories: $\{1,2,3\}$
\item
Markov dependance with transition probabilities:
\begin{align}
& \text{logit}(P(y<=1 | yp=1)) = a11 \nonumber \\
& \text{logit}(P(y<=2 | yp=1)) = a11 + a12\nonumber \\
& \text{logit}(P(y<=1 | yp=2)) = a21\nonumber \\
& \text{logit}(P(y<=2 | yp=2)) = a21 + a22\nonumber \\
& \text{logit}(P(y<=1 | yp=3)) = a31\nonumber \\
& \text{logit}(P(y<=2 | yp=3)) = a31 + a32\nonumber 
\end{align}
\end{itemize}


\begin{figure}[htbp!]
\includegraphics[width=.75\textwidth]{pics/cumulLogitMarkov} 
\caption{Markov transition probabilities for three-states model.}
\label{fig:cumulLogitMarkov}
\end{figure}

\subsubsection{NM-TRAN code:}

\myStartLine

\lstset{language=NONMEMdataSet}
\begin{lstlisting}
	missing
\end{lstlisting}

\myEndLine

\subsubsection{MLXTRAN code from Monolix 4.1 User Manual:}

\myStartLine

\begin{lstlisting}
DESCRIPTION:
Categorical data model with Markovian dependence,
3 categories - logit-probabilities

INPUT:
parameter = {a11, a12, a13, a21, a22, a23, a31, a32, a33}

OBSERVATION:
y = {
      type = categorical
      categories = {1,2,3}
      dependence = Markov
      logit(P(y<=1 | yp=1)) = a11
      logit(P(y<=2 | yp=1)) = a11 + a12
      logit(P(y<=1 | yp=2)) = a21
      logit(P(y<=2 | yp=2)) = a21 + a22
      logit(P(y<=1 | yp=3)) = a31
      logit(P(y<=2 | yp=3)) = a31 + a32
}
\end{lstlisting}

\myEndLine

\subsubsection{PharmML code:}
All parameters are assumed to be defined in the \xelem{ParameterModel} \emph{pm1}.

\lstset{language=XML}
\begin{lstlisting}
        <ObservationModel blkId="om1">
            <Discrete>
                <CategoricalData ordered="yes">
                    <SimpleParameter symbId="a11"/>
                    <!-- omitted a12...a32 -->
                    <SimpleParameter symbId="a33"/>
                    
                    <ListOfCategories> 
                        <Category symbId="cat1"/>
                        <Category symbId="cat2"/>
                        <Category symbId="cat3"/>
                    </ListOfCategories>
                    
                    <CategoryVariable symbId="y"/>
                    <PreviousStateVariable symbId="yp"/>
                    
                    <Dependance type="discreteMarkov"/>
                    
                    <!--   logit(P (y <= 1|yp = 1)) = a11 -->
                    <ProbabilityAssignment>
                        <Probability linkFunction="logit">
                            <CurrentState>
                                <math:LogicBinop op="leq">
                                    <ct:SymbRef symbIdRef="y"/>
                                    <ct:SymbRef symbIdRef="cat1"/>
                                </math:LogicBinop>
                            </CurrentState>
                            <PreviousState>
                                <math:LogicBinop op="eq">
                                    <ct:SymbRef symbIdRef="yp"/>
                                    <ct:SymbRef symbIdRef="cat1"/>
                                </math:LogicBinop>
                            </PreviousState>
                        </Probability>
                        <ct:Assign>
                            <ct:SymbRef symbIdRef="a11"/>
                        </ct:Assign>
                    </ProbabilityAssignment>
                    
                    <!--  logit(P (y <= 2|yp = 1)) = a11 + a12 -->
                    <ProbabilityAssignment>
                        <Probability linkFunction="logit">
                            <CurrentState>
                                <math:LogicBinop op="leq">
                                    <ct:SymbRef symbIdRef="y"/>
                                    <ct:SymbRef symbIdRef="cat2"/>
                                </math:LogicBinop>
                            </CurrentState>
                            <PreviousState>
                                <math:LogicBinop op="eq">
                                    <ct:SymbRef symbIdRef="yp"/>
                                    <ct:SymbRef symbIdRef="cat1"/>
                                </math:LogicBinop>
                            </PreviousState>
                        </Probability>
                        <ct:Assign>
                            <math:Equation>
                                <math:Binop op="plus">
                                    <ct:SymbRef symbIdRef="a11"/>
                                    <ct:SymbRef symbIdRef="a12"/>
                                </math:Binop>
                            </math:Equation>
                        </ct:Assign>
                    </ProbabilityAssignment>
                    <!--  logit(P (y <= 1|yp = 2)) = a21 -->
                    <!-- omitted here -->
                    <!--  logit(P (y <= 2|yp = 2)) = a21 + a22 -->
                    <!-- omitted here -->
                    <!--  logit(P (y <= 1|yp = 3)) = a31 -->
                    <!-- omitted here -->
                    <!--  logit(P (y <= 2|yp = 3)) = a31 + a32 -->
                    <!-- omitted here -->
                </CategoricalData>
            </Discrete>
        </ObservationModel>
\end{lstlisting}



%%%%%%%%%%%%%%%%%%%%%%%%%%%%%%%%%%%%%%%%%%%%%%%%%%%%%%%%%%%%%%%%
\subsection{Cumulative logit probabilities with Markovian and initial probabilities}
Based on an example described in \cite{MLXTRANforMonolix:2014}. 
This model extends the previous by adding the initial probabilities for a given category.

\paragraph{Observation model}

\begin{itemize}
\item
Type of observed variable -- discrete / ordered categorical
\item
Category variable: $y$
\item
Initial state variable: $yinit$
\item
Previous state variable: $yp$
\item
Set of categories: $\{1,2,3\}$
\item
Initial probabilities
\begin{align}
& P(y=1) = a1 \nonumber \\
& P(y=2) = a2  \nonumber
\end{align}
\item
Markov dependance with transition probabilities (with $y$ -- current state, $yp$ -- previous state):
\begin{align}
& \text{logit}(P(y<=1 | yp=1)) = a11 \nonumber \\
& \text{logit}(P(y<=2 | yp=1)) = a11 + a12\nonumber \\
& \text{logit}(P(y<=1 | yp=2)) = a21\nonumber \\
& \text{logit}(P(y<=2 | yp=2)) = a21 + a22\nonumber \\
& \text{logit}(P(y<=1 | yp=3)) = a31\nonumber \\
& \text{logit}(P(y<=2 | yp=3)) = a31 + a32\nonumber 
\end{align}
\end{itemize}

\subsubsection{NM-TRAN code:}

\myStartLine

\lstset{language=NONMEMdataSet}
\begin{lstlisting}
	missing
\end{lstlisting}

\myEndLine

\subsubsection{MLXTRAN code:}

\myStartLine

\lstset{language=MLXTRANcode}
\begin{lstlisting}
DESCRIPTION:
Cumulative ordered with Markov and initial probabilities

OBSERVATION:
State = {
	type = categorical
	categories = {1,2,3}
	dependence = Markov
	P(yinit=1) = a1
	P(yinit=2) = a2
	logit(P(y<=1|yp=1)) = a11
	logit(P(y<=2|yp=1)) = a11+a12
	logit(P(y<=1|yp=2)) = a21
	logit(P(y<=2|yp=2)) = a21+a22
	logit(P(y<=1|yp=3)) = a31
	logit(P(y<=2|yp=3)) = a31+a32
}
\end{lstlisting}

\myEndLine

\subsubsection{PharmML code:}
All parameters are assumed to be defined in the \xelem{ParameterModel} \emph{pm1}.

\lstset{language=XML}
\begin{lstlisting}
        <ObservationModel blkId="om1">
            <Discrete>
                <CategoricalData ordered="yes">
                    <SimpleParameter symbId="a11"/>
                    <!-- omitted a12...a32 -->
                    <SimpleParameter symbId="a33"/>
                    
                    <ListOfCategories> 
                        <Category symbId="cat1"/>
                        <Category symbId="cat2"/>
                        <Category symbId="cat3"/>
                    </ListOfCategories>
                    
                    <CategoryVariable symbId="y"/>
                    <InitialStateVariable symbId="yinit"/>
                    <PreviousStateVariable symbId="yp"/>
                    
                    <Dependance type="discreteMarkov"/>
                    
                    <!-- P(y = 1) = a1 -->
                    <ProbabilityAssignment>
                        <Probability linkFunction="identity">
                            <math:LogicBinop op="eq">
                                <ct:SymbRef symbIdRef="yinit"/>
                                <ct:SymbRef symbIdRef="cat1"/>
                            </math:LogicBinop>
                        </Probability>
                        <ct:Assign>
                            <ct:SymbRef symbIdRef="a1"/>
                        </ct:Assign>
                    </ProbabilityAssignment>
                    
                    <!-- P(y = 2) = a2 -->
                    <ProbabilityAssignment>
                        <Probability linkFunction="identity">
                            <math:LogicBinop op="eq">
                                <ct:SymbRef symbIdRef="yinit"/>
                                <ct:SymbRef symbIdRef="cat2"/>
                            </math:LogicBinop>
                        </Probability>
                        <ct:Assign>
                            <ct:SymbRef symbIdRef="a2"/>
                        </ct:Assign>
                    </ProbabilityAssignment>
                    
                    <!--   logit(P(y <= 1|yp = 1)) = a11 -->
                    <ProbabilityAssignment>
                        <!-- omitted here, identical as in previous example -->
                    </ProbabilityAssignment>
                    
                    <!--  logit(P(y <= 2|yp = 1)) = a11 + a12 -->
                    <ProbabilityAssignment>
                        <!-- omitted here, identical as in previous example -->
                    </ProbabilityAssignment>
                    <!--  logit(P(y <= 1|yp = 2)) = a21 -->
                    <!-- omitted here -->
                    <!--  logit(P(y <= 2|yp = 2)) = a21 + a22 -->
                    <!-- omitted here -->
                    <!--  logit(P(y <= 1|yp = 3)) = a31 -->
                    <!-- omitted here -->
                    <!--  logit(P(y <= 2|yp = 3)) = a31 + a32 -->
                    <!-- omitted here -->
                </CategoricalData>
            </Discrete>
        </ObservationModel>
\end{lstlisting}


%%%%%%%%%%%%%%%%%%%%%%%%%%%%%%%%%%%%%%%%%%%%%%%%%%%%%%%%%%%%%%%%
\subsection{Categorical with 2nd order Markovian dependence}

\paragraph{Observation model}

\begin{itemize}
\item
Type of observed variable -- discrete / ordered categorical
\item
Category variable: $y$
\item
Initial state variable: $yinit$
\item
Previous state variable: $yp$
\item
Set of categories: $\{1,2,3\}$
\item
Markov dependance with transition probabilities (with $y$ -- current state, $yp1$ -- 1st order state, $yp2$ -- 2nd order state):
\begin{align}
& logit(P(y<=1 | yp1=1, yp2=2)) = a112 \nonumber \\
& logit(P(y<=2 | yp1=1, yp2=3)) = a213 \nonumber \\
& logit(P(y<=1 | yp1=2, yp2=1)) = a121 \nonumber \\
& logit(P(y<=2 | yp1=2, yp2=3)) = a223 \nonumber \\
& logit(P(y<=1 | yp1=3, yp2=1)) = a131 \nonumber \\
& logit(P(y<=2 | yp1=3, yp2=2)) = a232 \nonumber 
\end{align}
\end{itemize}

\subsubsection{NM-TRAN code:}

\myStartLine

\lstset{language=NONMEMdataSet}
\begin{lstlisting}
missing
\end{lstlisting}

\myEndLine

\subsubsection{MLXTRAN code:}

\myStartLine

\lstset{language=MLXTRANcode}
\begin{lstlisting}
missing
\end{lstlisting}

\myEndLine

\subsubsection{PharmML code -- Version 1:}

\lstset{language=XML}
\begin{lstlisting}
                <CategoricalData ordered="yes">
                    <SimpleParameter symbId="a112"/>
                    <!-- omitted here a213...a131 -->
                    <SimpleParameter symbId="a232"/>
                    
                    <ListOfCategories> 
                        <Category symbId="cat1"/>
                        <Category symbId="cat2"/>
                        <Category symbId="cat3"/>
                    </ListOfCategories>
                    
                    <CategoryVariable symbId="y"/>
                    <PreviousStateVariable symbId="yp1"/>
                    <PreviousStateVariable symbId="yp2"/>
                    
                    <Dependance type="discreteMarkov"/>
                    
                    <!--   logit(P(y <= 1 | yp1 = 1, yp2 = 2)) = a112 -->
                    <ProbabilityAssignment>
                        <Probability linkFunction="logit">
                            <CurrentState>
                                <math:LogicBinop op="leq">
                                    <ct:SymbRef symbIdRef="y"/>
                                    <ct:SymbRef symbIdRef="cat1"/>
                                </math:LogicBinop>
                            </CurrentState>
                            <PreviousState MarkovOrder="1">
                                <math:LogicBinop op="eq">
                                    <ct:SymbRef symbIdRef="yp1"/>
                                    <ct:SymbRef symbIdRef="cat1"/>
                                </math:LogicBinop>
                            </PreviousState>
                            <PreviousState MarkovOrder="2">
                                <math:LogicBinop op="eq">
                                    <ct:SymbRef symbIdRef="yp2"/>
                                    <ct:SymbRef symbIdRef="cat2"/>
                                </math:LogicBinop>
                            </PreviousState>
                        </Probability>
                        <ct:Assign>
                            <ct:SymbRef symbIdRef="a112"/>
                        </ct:Assign>
                    </ProbabilityAssignment>
                    
                    <!--   logit(P(y<=2 | yp1=1, yp2=3)) = a213 -->
                    <ProbabilityAssignment>
                        <Probability linkFunction="logit">
                            <CurrentState>
                                <math:LogicBinop op="leq">
                                    <ct:SymbRef symbIdRef="y"/>
                                    <ct:SymbRef symbIdRef="cat2"/>
                                </math:LogicBinop>
                            </CurrentState>
                            <PreviousState MarkovOrder="1">
                                <math:LogicBinop op="eq">
                                    <ct:SymbRef symbIdRef="yp1"/>
                                    <ct:SymbRef symbIdRef="cat1"/>
                                </math:LogicBinop>
                            </PreviousState>
                            <PreviousState MarkovOrder="2">
                                <math:LogicBinop op="eq">
                                    <ct:SymbRef symbIdRef="yp2"/>
                                    <ct:SymbRef symbIdRef="cat3"/>
                                </math:LogicBinop>
                            </PreviousState>
                        </Probability>
                        <ct:Assign>
                            <ct:SymbRef symbIdRef="a213"/>
                        </ct:Assign>
                    </ProbabilityAssignment>
                    <!-- omitted here -->
                    <!-- logit(P (y <= 1|yp1 = 2, yp2 = 1)) = a121 
                         logit(P (y <= 2|yp1 = 2, yp2 = 3)) = a223 
                         logit(P (y <= 1|yp1 = 3, yp2 = 1)) = a131 
                         logit(P (y <= 2|yp1 = 3, yp2 = 2)) = a232 -->
                </CategoricalData>            
 \end{lstlisting}


\subsubsection{PharmML code -- Version 2:}
This version was proposed by Roberto Bizzotto. 
The \xelem{Condition} tag is used to define the Markovian dependences
instead of using \xelem{PreviousState} elements.

\lstset{language=XML}
\begin{lstlisting}
        <ObservationModel blkId="om2">
            <Discrete>
                <CategoricalData ordered="yes">
                    <SimpleParameter symbId="a112"/>
                    <!-- omitted here a213...a131 -->
                    <SimpleParameter symbId="a232"/>
                    
                    <ListOfCategories> 
                        <Category symbId="cat1"/>
                        <Category symbId="cat2"/>
                        <Category symbId="cat3"/>
                    </ListOfCategories>
                    
                    <CategoryVariable symbId="y"/>
                    <InitialStateVariable symbId ="yinit" />
                    <PreviousStateVariable symbId="yp1"/>
                    <PreviousStateVariable symbId="yp2"/>
                    
                    <Dependance type="discreteMarkov"/>

                    <!-- logit (P(y <= 1 | yp1 = 1, yp2 = 2)) = a112 -->
                    <ProbabilityAssignment>
                        <Probability linkFunction="identity">
                            <CurrentState>
                                <math:LogicBinop op="leq">
                                    <ct:SymbRef symbIdRef ="y" />
                                    <ct:SymbRef symbIdRef ="cat1" />
                                </math:LogicBinop >
                            </CurrentState>
                            <!-- Condition tag is used to define the Markovian dependences -->
                            <Condition>
                                <math:LogicBinop op="and">
                                    <math:LogicBinop op="eq">
                                        <ct:SymbRef symbIdRef ="yp1" />
                                        <ct:SymbRef symbIdRef ="cat1" />
                                    </math:LogicBinop >
                                    <math:LogicBinop op="eq">
                                        <ct:SymbRef symbIdRef ="yp2" />
                                        <ct:SymbRef symbIdRef ="cat2" />
                                    </math:LogicBinop >
                                </math:LogicBinop >
                            </Condition>
                        </Probability>
                        <ct:Assign>
                            <ct:SymbRef symbIdRef ="a112" />
                        </ct:Assign>
                    </ProbabilityAssignment>
                    
                    <!-- omitted here all other equations, as above-->
                    
                </CategoricalData>
            </Discrete>
        </ObservationModel>     
\end{lstlisting}

%%%%%%%%%%%%%%%%%%%%%%%%%%%%%%%%%%%%%%%%%%%%%%%%%%%%%%%%%%%%%%%%
\subsection{Categorical with continues Markovian dependence}
Based on an example described in \cite{MLXTRANforMonolix:2014}. 

\paragraph{Observation model}

\begin{itemize}
\item
Type of observed variable -- discrete / ordered categorical
\item
Category variable: $y$
\item
Initial state variable: $yinit$
\item
Previous state variable: $yp$
\item
Set of categories: $\{1,2\}$
\item
Initial probabilities
\begin{align}
& P(y=1) = p1 \nonumber
\end{align}
\item
Markov dependance with transition probabilities:
\begin{align}
& transitionRate(1,2) = q12  \nonumber \\
& transitionRate(2,1) = q21 \nonumber 
\end{align}
\end{itemize}

\subsubsection{NM-TRAN code:}

\myStartLine

\lstset{language=NONMEMdataSet}
\begin{lstlisting}
	missing
\end{lstlisting}

\myEndLine

\subsubsection{MLXTRAN code:}

\myStartLine

\lstset{language=MLXTRANcode}
\begin{lstlisting}
OBSERVATION:
State = {
	type = categorical
	categories = {1,2}
	dependence = Markov
	P(State_1=1) = p1
	transitionRate(1,2) = q12
	transitionRate(2,1) = q21
}
\end{lstlisting}

\myEndLine

\subsubsection{PharmML code:}

\lstset{language=XML}
\begin{lstlisting}
        <ObservationModel blkId="om1">
            <Discrete>
                <CategoricalData ordered="no">
                    <SimpleParameter symbId="p1"/>
                    <SimpleParameter symbId="q12"/>
                    <SimpleParameter symbId="q21"/>
                    
                    <ListOfCategories>
                        <Category symbId="cat1"/>
                        <Category symbId="cat2"/>
                    </ListOfCategories>
                    
                    <CategoryVariable symbId="y"/>
                    <InitialStateVariable symbId="yinit"/>
                    
                    <Dependance type="continuousMarkov"/>
                    
                    <!-- P(y = 1) = p1 -->
                    <ProbabilityAssignment>
                        <Probability>
                            <math:LogicBinop op="eq">
                                <ct:SymbRef symbIdRef="yinit"/>
                                <ct:SymbRef symbIdRef="cat1"/>
                            </math:LogicBinop>
                        </Probability>
                        <ct:Assign>
                            <ct:SymbRef symbIdRef="p1"/>
                        </ct:Assign>
                    </ProbabilityAssignment>
                    
                    <!-- transitionRate(1, 2) = q12 -->
                    <ProbabilityAssignment>
                        <TransitionRate>
                            <CurrentState> 
                                <math:LogicBinop op="eq">
                                    <ct:SymbRef symbIdRef="y"/>
                                    <ct:SymbRef symbIdRef="cat1"/>
                                </math:LogicBinop>
                            </CurrentState>
                            <PreviousState>
                                <math:LogicBinop op="eq">
                                    <ct:SymbRef symbIdRef="yp"/>
                                    <ct:SymbRef symbIdRef="cat2"/>
                                </math:LogicBinop>
                            </PreviousState>
                        </TransitionRate>
                        <ct:Assign>
                            <ct:SymbRef symbIdRef="q12"/>
                        </ct:Assign>
                    </ProbabilityAssignment>
                    
                    <!--   transitionRate(2, 1) = q21 -->
                    <ProbabilityAssignment>
                        <TransitionRate>
                            <CurrentState> 
                                <math:LogicBinop op="eq">
                                    <ct:SymbRef symbIdRef="y"/>
                                    <ct:SymbRef symbIdRef="cat2"/>
                                </math:LogicBinop>
                            </CurrentState>
                            <PreviousState>
                                <math:LogicBinop op="eq">
                                    <ct:SymbRef symbIdRef="yp"/>
                                    <ct:SymbRef symbIdRef="cat1"/>
                                </math:LogicBinop>
                            </PreviousState>
                        </TransitionRate>
                        <ct:Assign>
                            <ct:SymbRef symbIdRef="q21"/>
                        </ct:Assign>
                    </ProbabilityAssignment>
                </CategoricalData>
            </Discrete>
        </ObservationModel>
\end{lstlisting}



%%%%%%%%%%%%%%%%%%%%%%%%%%%%%%%%%%%%%%%%%%%%%%%%%%%%%%%%%%%%%%%%
\subsection{Reference category logit model -- sleep model}
The model names is after \cite{Agresti:2002fk}.

\paragraph{Observation model}

\begin{itemize}
\item
Type of observed variable -- discrete / ordered categorical
\item
Category variable: $y$
\item
Set of categories: $\{0,1,2,3,4,5\}$
\item
Probabilities
\begin{align}
& P(Y=1) = \exp(G1)/(1+\exp(G1)+\exp(G2)+\exp(G3)) \nonumber \\
& P(Y=2) = \exp(G2)/(1+\exp(G1)+\exp(G2)+\exp(G3)) \nonumber \\
& P(Y=3) = 0 \nonumber \\
& P(Y=4) = 0 \nonumber \\
& P(Y=5) = \exp(G3)/(1+\exp(G1)+\exp(G2)+\exp(G3)) \nonumber
\end{align}
\end{itemize}

%Y = {
%	type = categorical
%	categories = {0, 1, 2, 3, 4, 5}
%	;P(Y=0) = 1/(1+exp(G1)+exp(G2)+exp(G3))
%	P(Y=1) = exp(G1)/(1+exp(G1)+exp(G2)+exp(G3))
%	P(Y=2) = exp(G2)/(1+exp(G1)+exp(G2)+exp(G3))
%	P(Y=3) = 0
%	P(Y=4) = 0
%	P(Y=5) = exp(G3)/(1+exp(G1)+exp(G2)+exp(G3))
%}



\subsubsection{NM-TRAN code provided by Roberto Bizzotto, \cite{Bizzotto:2011fk}:}

\myStartLine

\lstset{language=NONMEMdataSet}
\begin{lstlisting}
; Sleep model with multinomial logistic functions
; Transitions from awake state
; Categorical data model - 4 categories


removed multiple assignments and definitions

  PAW=1/(1+EXP(G1)+EXP(G2)+EXP(G3))
  P1=EXP(G1)/(1+EXP(G1)+EXP(G2)+EXP(G3))
  P2=EXP(G2)/(1+EXP(G1)+EXP(G2)+EXP(G3))
  P3=0
  PR=EXP(G3)/(1+EXP(G1)+EXP(G2)+EXP(G3))


  Y=0
  IF (STAG.EQ.0) Y=PAW
  IF (STAG.EQ.1) Y=P1
  IF (STAG.EQ.2) Y=P2
  IF (STAG.EQ.5) Y=PR

\end{lstlisting}

\myEndLine

\subsubsection{MLXTRAN translation by Roberto Bizzotto and Marc Lavielle:}
part 1 - 'project001.mlxtran'  -- omitted \\
part 2 - 'trAW004'

\myStartLine

\lstset{language=MLXTRANcode}
\begin{lstlisting}
DESCRIPTION:
Sleep model with multinomial logistic functions
Transitions from awake state
Categorical data model - 4 categories

if (t>=BPA&&t=<BPB&&SL==1)
	G1=G1A*(BPB-t)/(BPB-BPA)+G1B*(t-BPA)/(BPB-BPA)
elseif (t>BPB&&t=<BPC&&SL==1)
	...
end


OBSERVATION:
Y = {
	type = categorical
	categories = {0, 1, 2, 3, 4, 5}
	;P(Y=0) = 1/(1+exp(G1)+exp(G2)+exp(G3))
	P(Y=1) = exp(G1)/(1+exp(G1)+exp(G2)+exp(G3))
	P(Y=2) = exp(G2)/(1+exp(G1)+exp(G2)+exp(G3))
	P(Y=3) = 0
	P(Y=4) = 0
	P(Y=5) = exp(G3)/(1+exp(G1)+exp(G2)+exp(G3))
}

OUTPUT:
output = Y
\end{lstlisting}

\myEndLine

\subsubsection{PharmML code:}
All parameters are assumed to be defined in the \xelem{ParameterModel} \emph{pm1}.

\lstset{language=XML}
\begin{lstlisting}
        <ObservationModel blkId="om1">
            <Discrete>
                <CategoricalData ordered="no">

                    <ListOfCategories> 
                        <Category symbId="cat0"/>
                        <!-- omitted cat1...cat4 -->
                        <Category symbId="cat5"/>
                    </ListOfCategories>
                    
                    <CategoryVariable symbId="Y"/>
                    
                    <!-- P(Y = 1) = exp(G1)/(1+exp(G1)+exp(G2)+exp(G3)) --> 
                    <ProbabilityAssignment>
                        <Probability linkFunction="identity">
                            <math:LogicBinop op="eq">
                                <ct:SymbRef symbIdRef="Y"/>
                                <ct:SymbRef symbIdRef="cat1"/>
                            </math:LogicBinop>
                        </Probability>
                        <ct:Assign>
                            <math:Equation>
                                <math:Binop op="divide">
                                    <math:Uniop op="exp">
                                        <ct:SymbRef symbIdRef="G1"/>
                                    </math:Uniop>
                                    <math:Binop op="plus">
                                        <math:Binop op="plus">
                                            <math:Binop op="plus">
                                                <ct:Real>1</ct:Real>
                                                <math:Uniop op="exp">
                                                    <ct:SymbRef symbIdRef="G1"/>
                                                </math:Uniop>
                                            </math:Binop>
                                            <math:Uniop op="exp">
                                                <ct:SymbRef symbIdRef="G2"/>
                                            </math:Uniop>
                                        </math:Binop>
                                        <math:Uniop op="exp">
                                            <ct:SymbRef symbIdRef="G3"/>
                                        </math:Uniop>
                                    </math:Binop>
                                </math:Binop>
                            </math:Equation>
                        </ct:Assign>
                    </ProbabilityAssignment>
                    
                    <!-- P(Y = 2) = exp(G2)/(1+exp(G1)+exp(G2)+exp(G3)) --> 
                    <ProbabilityAssignment>
                        <Probability linkFunction="identity">
                            <math:LogicBinop op="eq">
                                <ct:SymbRef symbIdRef="Y"/>
                                <ct:SymbRef symbIdRef="cat1"/>
                            </math:LogicBinop>
                        </Probability>
                        <ct:Assign>
                            <math:Equation>
                                <math:Binop op="divide">
                                    <math:Uniop op="exp">
                                        <ct:SymbRef symbIdRef="G2"/>
                                    </math:Uniop>
                                    <math:Binop op="plus">
                                        <math:Binop op="plus">
                                            <math:Binop op="plus">
                                                <ct:Real>1</ct:Real>
                                                <math:Uniop op="exp">
                                                    <ct:SymbRef symbIdRef="G1"/>
                                                </math:Uniop>
                                            </math:Binop>
                                            <math:Uniop op="exp">
                                                <ct:SymbRef symbIdRef="G2"/>
                                            </math:Uniop>
                                        </math:Binop>
                                        <math:Uniop op="exp">
                                            <ct:SymbRef symbIdRef="G3"/>
                                        </math:Uniop>
                                    </math:Binop>
                                </math:Binop>
                            </math:Equation>
                        </ct:Assign>
                    </ProbabilityAssignment>
                    
                    <!-- P(Y = 3) = 0 -->
                    <ProbabilityAssignment>
                        <Probability linkFunction="identity">
                            <math:LogicBinop op="eq">
                                <ct:SymbRef symbIdRef="Y"/>
                                <ct:SymbRef symbIdRef="cat3"/>
                            </math:LogicBinop>
                        </Probability>
                        <ct:Assign>
                            <ct:Real>0</ct:Real>
                        </ct:Assign>
                    </ProbabilityAssignment>
                    
                    <!-- P(Y = 4) = 0 --> 
                    <ProbabilityAssignment>
                        <Probability linkFunction="identity">
                            <math:LogicBinop op="eq">
                                <ct:SymbRef symbIdRef="Y"/>
                                <ct:SymbRef symbIdRef="cat4"/>
                            </math:LogicBinop>
                        </Probability>
                        <ct:Assign>
                            <ct:Real>0</ct:Real>
                        </ct:Assign>
                    </ProbabilityAssignment>
                    
                    <!-- P(Y = 5) = exp(G5)/(1+exp(G1)+exp(G2)+exp(G3)) --> 
                    <ProbabilityAssignment>
                        <Probability linkFunction="identity">
                            <math:LogicBinop op="eq">
                                <ct:SymbRef symbIdRef="Y"/>
                                <ct:SymbRef symbIdRef="cat5"/>
                            </math:LogicBinop>
                        </Probability>
                        <ct:Assign>
                            <math:Equation>
                                <math:Binop op="divide">
                                    <math:Uniop op="exp">
                                        <ct:SymbRef symbIdRef="G5"/>
                                    </math:Uniop>
                                    <math:Binop op="plus">
                                        <math:Binop op="plus">
                                            <math:Binop op="plus">
                                                <ct:Real>1</ct:Real>
                                                <math:Uniop op="exp">
                                                    <ct:SymbRef symbIdRef="G1"/>
                                                </math:Uniop>
                                            </math:Binop>
                                            <math:Uniop op="exp">
                                                <ct:SymbRef symbIdRef="G2"/>
                                            </math:Uniop>
                                        </math:Binop>
                                        <math:Uniop op="exp">
                                            <ct:SymbRef symbIdRef="G3"/>
                                        </math:Uniop>
                                    </math:Binop>
                                </math:Binop>
                            </math:Equation>
                        </ct:Assign>
                    </ProbabilityAssignment>
                </CategoricalData>
            </Discrete>
        </ObservationModel>
\end{lstlisting}


%%%%%%%%%%%%%%%%%%%%%%%%%%%%%%%%%%%%%%%%%%%%%%%%%%%%%%%%%%%%%%%%
\subsection{PK with categorical effect 1 {\color{red} \scshape{*}}}
\label{subsec:PKPDcategorical}

%%%%%%%%%%%%%%%%%%%%%%%%%%%%%%%%%%%%%%%%%%%%%%%%%%%%%%%%%%%%%%%%%
%\subsubsection{Introduction}
%\label{subsec:exp4_intro}
%
This example is based on the document \cite{Lavielle:2011}. The essential element of this 
task is the probability distribution of categorical data Y, which can be defined as either 
$P(Y=k)$ or $\log(P(Y=k))$ or $logit(P(Y=t))$. In this example, $k \in {1,2}$, i.e. the effect 
outcome is either 1 or 2. The underlying PK model is 1-compartmental oral model. 
The probability for outcome is '1' is given by the equation
\begin{eqnarray}
	P(Y=1) &=& \frac{1}{1+\exp(-\theta_1 - \theta_2 \log(Cc))} 
\end{eqnarray}
which is plotted in Figure \ref{fig:p1surface}. This probability is represented as a function 
of $\theta_1$, $\theta_2$ and $log(Cc)$ which is visualised for three different values of $Cc = \{1,5,15\}$. 
%To get a full picture of the $p1$ dynamics run the Matlab animation using the code provided in \ref{MATLAB_p1animation}.

\begin{figure}[htbp]
\begin{center}
\includegraphics[width=.45\textwidth]{pics/CTS4_p1_threeSurfaces.png}
\caption{p1-probability surface as function of $\theta_1$ and $\theta_2$ plotted for Cc = \{1,5,15\} ($1\equiv$ green, $5 \equiv$ red, $15 \equiv$ blue). }
\label{fig:p1surface}
\end{center}
\end{figure}
%
%%%%%%%%%%%%%%%%%%%%%%%%%%%%%%%%%%%%%%%%%%%%%%%%%%%%%%%%%%%%%%%%%
%\subsubsection{Task description}
%\label{subsec:exp4_TaskDescription}
%
%\paragraph{Trial execution model} %% �design
%-- There are 4 arms (study groups of patients) i.e. 
%\begin{eqnarray}
%&&SizeArm=\{20, 20, 40, 40\}  \nonumber
%\end{eqnarray}
%
%-- Dosing depends on the arm. For example, in the first arm, dosing starts on $0h$ and repeats every $24 h$
%\begin{eqnarray}
%&&DoseTime=\{0:24:192, 0:48:192, 0:24:192, 0:48:192\}   \nonumber 
%\end{eqnarray}
%
%-- Dosing depends on the arm too. It is adjusted to body weight, e.g. $1 mg$ per $1 kg$ of body weight
%\begin{eqnarray}
%&&DoseSize = \{0.25, 0.5, 0.5, 1\} \nonumber \\
%&&DosePerKg = yes  \nonumber
%\end{eqnarray}
%
%-- Time of measurement for PK and PD happens according to different schedules, here $ObservationTime\{1\}$ and $ObservationTime\{2\}$ correspond to PK and PD, accordingly
%\begin{eqnarray}
%&&ObservationTime\{1\}=[0.5, 4:4:48, 52:24:192,192:4:250]  \nonumber \\
%&&ObservationTime\{2\}=0:24:288  \nonumber
%\end{eqnarray}

%\paragraph{Individual parameters model}
%
%Details omitted here...
%\begin{eqnarray}
%ka& \sim& \mbox{logNormal}(pop_{ka}, \omega_{ka});  \quad pop_{ka}=1,\quad \omega_{ka}=0.6 \nonumber \\
%V& \sim& \mbox{logNormal}(pop_{V}, \omega_{V}); \quad pop_V=8,\quad \omega_V=0.2 \nonumber \\
%CL& \sim&  \mbox{logNormal}(pop_{CL}, \omega_{CL}); \quad pop_{CL}=0.13,\quad \omega_{CL}=0.2 \nonumber \\
%theta1& \sim&  \mbox{Normal}(pop_{theta1}, \omega_{theta1}); \quad pop_{theta1}=-1,\quad \omega_{theta1}=0.3 \nonumber \\
%theta2& \sim&  \mbox{logNormal}(pop_{theta2}, \omega_{theta2}); \quad pop_{theta2}=1,\quad \omega_{theta2}=0.2 \nonumber 
%\end{eqnarray}
%\begin{eqnarray}
%\log(V_i) &=& \log(V_{pop}) + \beta_{1,V}\log(W_i/70) + \eta_{V,i} \nonumber \\
%\log(CL_i) &=& \log(CL_{pop}) + \beta_{1,CL}\log(W_i/70) + \eta_{CL,i} \nonumber \\
%\beta_{1,V}&=&1 \nonumber \\
%\beta_{1,CL}&=&0.75 \nonumber \\
%\rho_{V,CL}&=&0.7 \mbox{  (the correlation coefficient between } \eta_{V,i} \mbox{ and } \eta_{CL,i}  \mbox{)} \nonumber
%\end{eqnarray}
%\paragraph{Variance-covariance matrix of the random effects}
%\[
% \Omega =
% \begin{pmatrix}
%  \omega_{ka}^2 	& \omega_{ka,V} 	& \omega_{ka,CL} 	& \omega_{ka,theta1} 	& \omega_{ka,theta2} \\
%   			  	& \omega_{V}^2	& \omega_{V,CL}	& \omega_{V,theta1} 	& \omega_{V,theta2} \\
%  				& 				& \omega_{CL}^2 	& \omega_{CL,theta1} 	& \omega_{CL,theta2} \\
%  				& 				& 				& \omega_{theta1}^2 	& \omega_{theta1,theta2} \\
%  				& 				& 				& 					& \omega_{theta2}^2
% \end{pmatrix}
%= 
%\begin{pmatrix}
%  \omega_{ka}^2 	& 0			 	& 0			 	& 0				 	& 0				 \\
%   			  	& \omega_{V}^2	& \omega_{V,CL}	& 0				 	& 0				 \\
%  				& 				& \omega_{CL}^2 	& 0				 	& 0				 \\
%  				& 				& 				& \omega_{theta1}^2 	& 0				 \\
%  				& 				& 				& 					& \omega_{theta2}^2
% \end{pmatrix}
%\]

%\paragraph{Covariate model}
%\begin{eqnarray}
%Covariates: && \log(Weight/70)  \nonumber \\
%CovariatesType: && Continuous \nonumber \\
%CovariatesFile: && e.g. 'warfarindata.txt' \nonumber
%\end{eqnarray}

\paragraph{Structural model}
Oral 1-compartmental model with 1$^{st}$ order absorption, $ka$, and linear elimination, $k=CL/V$, 
which reads
\begin{eqnarray}
\frac{dAd}{dt} &=&-ka\times Ad \nonumber \\
\frac{dAc}{dt}&=&ka\times Ad - k\times Ac \nonumber \\ 
Cc &=& Ac/V \nonumber 
\end{eqnarray}

\paragraph{Observation model}

\begin{itemize}
\item
Type of observed variable -- discrete/categorical
\item
Category variable: $Y$
\item
Set of categories: $\{0,1\}$
\item
Probability for the '1' category
\begin{eqnarray}
P(Y=1) &=& \frac{1}{1+\exp(-\theta1 - \theta2 \times \log(Cc))} \nonumber
\end{eqnarray}
\end{itemize}

\subsubsection{NM-TRAN code based on an use case by Nick Holford and Mike Smith:}

\myStartLine

\lstset{language=NONMEMdataSet}
\begin{lstlisting}
$PROB BINARY OUTCOME
$DATA binary.csv IGNORE #
...

$THETA 
(********PK model parameters omitted********)
(.01,.1,.99) ; BaseP
(0,.5,10) ; Beta

$OMEGA BLOCK(3) FIX
(********PK model parameters omitted********)
$OMEGA
0.04 ; ppv_event

(********PK model definition omitted********)

$ERROR
  IF (ICALL.EQ.4) CP=F

  BASE=LOG(THETA(4)/(1-THETA(4))) ; transform to logit
  LGST=BASE + THETA(5)*CP + ETA(4)
  P1=1/(1+EXP(-LGST) ; untransform from logit

(*********simulation code omitted*********)
...
\end{lstlisting}

\myEndLine

\subsubsection{MLXTRAN code by MJS:}

\myStartLine

\lstset{language=MLXTRANcode}
\begin{lstlisting}
DESCRIPTION: 
Joint PK and categorical data model 

INPUT: 
parameter = {ka, V, Cl, theta1, theta2} 

EQUATION: {Cc, Ce} = pkmodel(ka, V, Cl) 

OBSERVATION: 
Y = { 
	type = categorical 
	categories = {1, 2} 
	P(Y=1) = 1 / (1 + exp(-theta1 - theta2*Ce))
} 
\end{lstlisting}

\myEndLine

\subsubsection{PharmML code}
In the following code only the observation model for the binary outcome variable $Y$ will be considered.
All parameters are assumed to be defined in the \xelem{ParameterModel} \emph{pm1}.

\lstset{language=XML}
\begin{lstlisting}
        <ObservationModel blkId="om1">
            <Discrete>
                <CategoricalData ordered="no">
                    <ListOfCategories> 
                        <Category symbId="cat0"/>
                        <Category symbId="cat1"/>
                    </ListOfCategories>
                    
                    <CategoryVariable symbId="y"/>
                    
                    <!-- P(y = 1) = 1 / (1 + exp(-theta1 - theta2*log(Cc))) -->
                    <ProbabilityAssignment>
                        <Probability>
                            <math:LogicBinop op="eq">
                                <ct:SymbRef symbIdRef="y"/>
                                <ct:SymbRef symbIdRef="cat1"/>
                            </math:LogicBinop>
                        </Probability>
                        <ct:Assign>
                            <math:Equation>
                                <math:Binop op="divide">
                                    <ct:Real>1</ct:Real>
                                    <math:Binop op="plus">
                                        <ct:Real>1</ct:Real>
                                        <math:Uniop op="exp">
                                            <math:Binop op="minus">
                                                <math:Uniop op="minus">
                                                    <ct:SymbRef blkIdRef="pm1" symbIdRef="theta1"/>
                                                </math:Uniop>
                                                <math:Binop op="times">
                                                    <ct:SymbRef blkIdRef="pm1" symbIdRef="theta2"/>
                                                    <math:Uniop op="log">
                                                        <ct:SymbRef blkIdRef="sm1" symbIdRef="Cc"/>
                                                    </math:Uniop>
                                                </math:Binop>
                                            </math:Binop>
                                        </math:Uniop>
                                    </math:Binop>
                                </math:Binop>
                            </math:Equation>
                        </ct:Assign>
                    </ProbabilityAssignment>
                </CategoricalData>
            </Discrete>
        </ObservationModel>
\end{lstlisting}


%\subsubsection{MATLAB code example}
%See \ref{subsec:MATLAB_categorical} for an example and instructions how to run it in MATLAB.

\subsection{PK with categorical effect 2 {\color{red} \scshape{new}}}
\label{subsec:PKPDcategorical2}

\paragraph{Structural model}
Oral 1-compartmental model with 1$^{st}$ order absorption, $ka$, and linear elimination, $k=CL/V$. 
%which reads
%\begin{table}[h!]
%\setlength{\tabcolsep}{15pt}
%\begin{center}
%\begin{tabular}{l}
%  \hline \hline
%PK macro  \\[-.25ex]
%  \hline
%\lstset{language=NONMEMdataSet}
%\begin{lstlisting}
%input = {ka, V, CL, V}
%PK:
%compartment(cmt=1, concentration=C, volume=V)
%oral(adm=1, cmt=1, ka)
%elimination(cmt=1, k=CL/V)
%\end{lstlisting}
%\\
%  \hline
%\end{tabular}
%\caption{PK macros for the model (corresponding to the ADVAN2 TRANS2).}
%\label{tab:advan2Table}
%\end{center}
%\end{table}

\paragraph{Observation model}

\begin{itemize}
\item
Type of observed variable -- discrete/categorical
\item
Category variable: $Y$
\item
Set of categories: $\{0,1,2,3\}$
\item
Probabilities 
\begin{eqnarray}
logit(P(Y<=0)) &=& EDRUG + B0 \nonumber \\
logit(P(Y<=1)) &=& EDRUG + B0 + B1 \nonumber \\
logit(P(Y<=2)) &=& EDRUG + B0 + B1 + B2 \nonumber
\end{eqnarray}
\end{itemize}

\subsubsection{NM-TRAN code:}
\myStartLine

\lstset{language=NONMEMdataSet}
\begin{lstlisting}
# A0 ... A2 cumul. logits
A0 = EDRUG + B0
A1 = EDRUG + B0 + B1
A2 = EDRUG + B0 + B1+ B2
 
# CP0 ... CP2 cumul. prob.
CP0 = 1/(1+exp(-A0))
CP1 = 1/(1+exp(-A1))
CP2 = 1/(1+exp(-A2))
 
# P0 ... P3 exact probabs
P0 = CP0
P1 = CP1 - CP0
P2 = CP2 - CP1
P3 = 1 - CP2
\end{lstlisting}

\myEndLine

\subsubsection{MLXTRAN code by MJS:}

\myStartLine

\lstset{language=MLXTRANcode}
\begin{lstlisting}
DESCRIPTION: 
Joint PK and categorical data model 

INPUT: 
parameter = {POP_BETA, B0, B1, B2} 

EQUATION: EDRUG = Cc * POP_BETA

OBSERVATION: 
y = {
  type = categorical
  categories = {0, 1, 2, 3}
  logit(P(y<=0)) = EDRUG + B0
  logit(P(y<=1)) = EDRUG + B0 + B1
  logit(P(y<=2)) = EDRUG + B0 + B1+ B2}
}
\end{lstlisting}

\myEndLine

\subsubsection{PharmML code -- with 2 alternative implementations of the \xelem{ObservationModel} -- declaration and assignment based}
\lstset{language=XML}
\begin{lstlisting}
<PharmML xmlns:xsi="http://www.w3.org/2001/XMLSchema-instance"
	...
    xmlns:mml="http://www.pharmml.org/2013/03/PharmML"
    implementedBy="MJS" writtenVersion="0.4" id="i1">

    <IndependentVariable symbId="t"/>
    
    <ModelDefinition id="i3" xmlns="http://www.pharmml.org/2013/03/ModelDefinition">
        
        <ParameterModel blkId="pm1">
            <SimpleParameter symbId="ka"/>
            <SimpleParameter symbId="V"/>
            <SimpleParameter symbId="CL"/>
            
            <SimpleParameter symbId="B0"/>
            <SimpleParameter symbId="B1"/>
            <SimpleParameter symbId="B2"/>
            
            <SimpleParameter symbId="POP_BETA"/>
        </ParameterModel>
        
        <StructuralModel blkId="sm1">
	   <!-- omitted details -->
        </StructuralModel>
        
        <!-- DECLARATION/MLXTRAN style -->
        <ObservationModel blkId="om1">
            <Discrete>
                <CategoricalData ordered="yes">
                    
                    <ListOfCategories> 
                        <Category symbId="cat0"/>
                        <Category symbId="cat1"/>
                        <Category symbId="cat2"/>
                        <Category symbId="cat3"/>
                    </ListOfCategories>
                    
                    <CategoryVariable symbId="y"/>
                    
                    <!-- logit( P(y <= 0) ) = EDRUG + B0 --> 
                    <ProbabilityAssignment>
                        <Probability linkFunction="logit">
                            <math:LogicBinop op="leq">
                                <ct:SymbRef symbIdRef="y"/>
                                <ct:SymbRef symbIdRef="cat0"/>
                            </math:LogicBinop>
                        </Probability>
                        <ct:Assign>
                            <math:Equation>
                                <math:Binop op="plus">
                                    <ct:SymbRef blkIdRef="sm1" symbIdRef="EDRUG"/>
                                    <ct:SymbRef blkIdRef="pm1" symbIdRef="B0"/>
                                </math:Binop>
                            </math:Equation>
                        </ct:Assign>
                    </ProbabilityAssignment>
                    
                    <!-- logit( P(y <= 1) ) = EDRUG + B0 + B1 --> 
                    <ProbabilityAssignment>
                        <Probability linkFunction="logit">
                            <math:LogicBinop op="leq">
                                <ct:SymbRef symbIdRef="y"/>
                                <ct:SymbRef symbIdRef="cat1"/>
                            </math:LogicBinop>
                        </Probability>
                        <ct:Assign>
                            <math:Equation>
                                <math:Binop op="plus">
                                    <ct:SymbRef blkIdRef="sm1" symbIdRef="EDRUG"/>
                                    <math:Binop op="plus">
                                        <ct:SymbRef symbIdRef="B0"/>
                                        <ct:SymbRef symbIdRef="B1"/>
                                    </math:Binop>
                                </math:Binop>
                            </math:Equation>
                        </ct:Assign>
                    </ProbabilityAssignment>
                    
                    <!-- logit( P(y <= 2) ) = EDRUG + B0 + B1 + B2 --> 
                    <ProbabilityAssignment>
                        <Probability linkFunction="logit">
                            <math:LogicBinop op="leq">
                                <ct:SymbRef symbIdRef="y"/>
                                <ct:SymbRef symbIdRef="cat2"/>
                            </math:LogicBinop>
                        </Probability>
                        <ct:Assign>
                            <math:Equation>
                                <math:Binop op="plus">
                                    <ct:SymbRef blkIdRef="sm1" symbIdRef="EDRUG"/>
                                    <math:Binop op="plus">
                                        <ct:SymbRef symbIdRef="B0"/>
                                        <math:Binop op="plus">
                                            <ct:SymbRef symbIdRef="B1"/>
                                            <ct:SymbRef symbIdRef="B2"/>
                                        </math:Binop>
                                    </math:Binop>
                                </math:Binop>
                            </math:Equation>
                        </ct:Assign>
                    </ProbabilityAssignment>
                </CategoricalData>
            </Discrete>
        </ObservationModel>
        
        <!-- ASSIGNMENT/NONMEM style -->
        <ObservationModel blkId="om2">
            <Discrete>
                <CategoricalData ordered="yes">
                    
                    <ct:Variable symbolType="real" symbId="A0">
                        <!-- EDURG + B0 -->
                    </ct:Variable>
                    <ct:Variable symbolType="real" symbId="A1">
                        <!-- EDURG + B0 + B1 -->
                    </ct:Variable>
                    <ct:Variable symbolType="real" symbId="A2">
                        <!-- EDURG + B0 + B1 + B2 -->
                    </ct:Variable>

                    <ct:Variable symbolType="real" symbId="CP0">
                        <!-- 1 / (1 + exp(-A0)) -->
                    </ct:Variable>
                    <ct:Variable symbolType="real" symbId="CP1">
                        <!-- 1 / (1 + exp(-A1)) -->
                    </ct:Variable>
                    <ct:Variable symbolType="real" symbId="CP2">
                        <!-- 1 / (1 + exp(-A2)) -->
                    </ct:Variable>

                    <ct:Variable symbolType="real" symbId="P0">
                        <!-- CP0 -->
                    </ct:Variable>
                    <ct:Variable symbolType="real" symbId="P1">
                        <!-- CP1 - CP0 -->
                    </ct:Variable>
                    <ct:Variable symbolType="real" symbId="P2">
                        <!-- CP2 - CP1 -->
                    </ct:Variable>
                    <ct:Variable symbolType="real" symbId="P3">
                        <!-- 1 - CP2 -->
                    </ct:Variable>
                    
                    <ListOfCategories> 
                        <Category symbId="cat0"/>
                        <Category symbId="cat1"/>
                        <Category symbId="cat2"/>
                        <Category symbId="cat3"/>
                    </ListOfCategories>
                    
                    <CategoryVariable symbId="y"/>
                    
                    <PMF linkFunction="identity">
                        <CategoricalDistribution xmlns="http://www.uncertml.org/3.0" 
                            definition="http://www.uncertml.org/...">
                            <categoryProb definition="http://www.uncertml.org/...">
                                <name>Probability for cat0</name>
                                <probability>
                                    <var varId="P0"/>
                                </probability>
                            </categoryProb>
                            <categoryProb definition="http://www.uncertml.org/...">
                                <name>Probability for cat1</name>
                                <probability>
                                    <var varId="P1"/>
                                </probability>
                            </categoryProb>
                            <categoryProb definition="http://www.uncertml.org/...">
                                <name>Probability for cat2</name>
                                <probability>
                                    <var varId="P2"/>
                                </probability>
                            </categoryProb>
                            <categoryProb definition="http://www.uncertml.org/...">
                                <name>Probability for cat3</name>
                                <probability>
                                    <var varId="P3"/>
                                </probability>
                            </categoryProb>
                        </CategoricalDistribution>
                    </PMF>
                </CategoricalData>
            </Discrete>
        </ObservationModel>
        
    </ModelDefinition>
    
    <mstep:ModellingSteps>
        
        <mstep:NONMEMdataSet oid="NMoid">
            
            <mstep:ColumnMapping>
                <ds:ColumnRef columnIdRef="TIME"/>
                <ct:SymbRef symbIdRef="t"/>
            </mstep:ColumnMapping>
            
            <mstep:ColumnMapping>
                <ds:ColumnRef columnIdRef="DV"/>
                <ct:SymbRef blkIdRef="om1" symbIdRef="y"/>
                <ds:CategoryMapping>
                    <ds:Map dataSymbol="0" modelSymbol="cat0"/>
                    <ds:Map dataSymbol="1" modelSymbol="cat1"/>
                    <ds:Map dataSymbol="2" modelSymbol="cat2"/>
                    <ds:Map dataSymbol="3" modelSymbol="cat3"/>
                </ds:CategoryMapping>
            </mstep:ColumnMapping>
            
            <ds:DataSet>
                <ds:Definition>
                    <ds:Column columnId="ID" columnType="id" valueType="id" columnNum="1"/>
                    <ds:Column columnId="TIME" columnType="time" valueType="real" columnNum="2"/>
                    <ds:Column columnId="DV" columnType="dv" valueType="real" columnNum="3"/>
                </ds:Definition>
                <ds:ImportData oid="importData">
                    <ds:path>myFile.csv</ds:path>
                    <ds:format>CSV</ds:format>
                    <ds:delimiter>COMMA</ds:delimiter>
                </ds:ImportData>
            </ds:DataSet>
        </mstep:NONMEMdataSet>
        
        <mstep:EstimationStep oid="estStep1">
            
            <mstep:TargetToolReference>
                <ct:OidRef oidRef="NMoid"/>
            </mstep:TargetToolReference>
            
            <mstep:ParametersToEstimate>
                <mstep:ParameterEstimation>
                    <ct:SymbRef symbIdRef="B0"/>
                    <mstep:InitialEstimate>
                        <ct:Real>1</ct:Real>
                    </mstep:InitialEstimate>
                </mstep:ParameterEstimation>
                
                <mstep:ParameterEstimation>
                    <ct:SymbRef symbIdRef="B1"/>
                    <mstep:InitialEstimate>
                        <ct:Real>0.6</ct:Real>
                    </mstep:InitialEstimate>
                </mstep:ParameterEstimation>
                
                <mstep:ParameterEstimation>
                    <ct:SymbRef symbIdRef="B2"/>
                    <mstep:InitialEstimate>
                        <ct:Real>0.6</ct:Real>
                    </mstep:InitialEstimate>
                </mstep:ParameterEstimation>
                
            </mstep:ParametersToEstimate>
            
            <mstep:Operation order="1" opType="estPop"/>
        </mstep:EstimationStep>
    </mstep:ModellingSteps>
</PharmML>

\end{lstlisting}





%%%%%%%%%%%%%%%%%%%%%%%%%%%%%%%%%%%%%%%%%%%%%%%%%%%%%%%%%%%%%%%%
\subsection{PK with continuous and categorical effect}
\label{subsec:PKPDcategorical2}
This model was suggested by Marc Lavielle in the \emph{Four models} document and combines 
both continuous and discrete observation models.


\paragraph{Model description -- PK}

\begin{itemize}
\item
Type of observed variable -- continuous
\item
Continuous variables: $Cc$ and $E$
\item
PK equations
\begin{align}
& \frac{dAd}{dt} = -ka \times Ad  \nonumber \\
& \frac{dAc}{dt} = ka \times Ad - k \times Ac  \nonumber \\
& Cc = \frac{Ac}{V}  \nonumber \\
& E = \frac{Emax \times Cc}{EC50 + Cc}  \nonumber
\end{align}
\end{itemize}


\paragraph{Observation model}

\begin{itemize}
\item
Type of observed variable -- discrete/categorical
\item
Category variable: $y$
\item
Set of categories: $\{1,2,3\}$
\item
Cumulative logits
\begin{align}
& logit(P(y<=1) )= lp1 = th1 + th2 \times Cc  \nonumber \\
& logit(P(y<=2) )= lp2 = th1 + th2 \times Cc + th3 \nonumber 
\end{align}
\end{itemize}

\subsubsection{NM-TRAN code:}

\myStartLine

\lstset{language=NONMEMdataSet}
\begin{lstlisting}
	missing
\end{lstlisting}

\myEndLine

\subsubsection{MLXTRAN code by Marc Lavielle:}

\myStartLine

\lstset{language=MLXTRANcode}
\begin{lstlisting}
[LONGITUDINAL]
input = {ka, V, k, a1, a2, Emax, EC50, th1, th2, th3}
PK:
depot(target=Ad)
EQUATION:
ddt_Ad = -ka*Ad
ddt_Ac = ka*Ad - k*Ac
Cc = Ac/V
E = Emax*Cc/(EC50 + Cc)
lp1 = th1 + th2*Cc
lp2 = th1 + th2*Cc + th3
DEFINITION:
Concentration = {distribution=lognormal, prediction=Cc, sd=a1}
Effect = {distribution=normal, prediction=E, sd=a2}
Level = {
	type=categorical, categories={1,2,3},
	logit(P(Level<=1)) = lp1
	logit(P(Level<=2)) = lp2
}
\end{lstlisting}

\myEndLine

\subsubsection{PharmML code:}
All parameters are assumed to be defined in the \xelem{ParameterModel} \xatt{pm1}.
Furthermore, it is assumed that the probabilities $lp1$ and $lp2$ are defined
in the structural model \xatt{sm1}.

\lstset{language=XML}
\begin{lstlisting}
        <ObservationModel blkId="om1">
            <ContinuousData>
                <RandomVariable symbId="eta_Cc">
                    <ct:VariabilityReference>
                        <ct:SymbRef symbIdRef="residual"/>
                    </ct:VariabilityReference>
                    <NormalDistribution xmlns="http://www.uncertml.org/3.0" 
                    definition="http://www.uncertml.org/3.0">
                        <mean>
                            <rVal>0</rVal>
                        </mean>
                        <stddev>
                            <prVal>1</prVal>
                        </stddev>
                    </NormalDistribution>
                </RandomVariable>
                <Standard symbId="C">
                    <Output>
                        <ct:SymbRef symbIdRef="C"/>
                    </Output>
                    <ErrorModel>
                        <ct:Assign>
                            <ct:SymbRef symbIdRef="a1"/>
                        </ct:Assign>
                    </ErrorModel>
                    <ResidualError>
                        <ct:SymbRef symbIdRef="epsilon_C"/>
                    </ResidualError>
                </Standard>
            </ContinuousData>
        </ObservationModel>
        
        <!-- Continuous model for E -->
        <ObservationModel blkId="om2">
            <ContinuousData>
                <RandomVariable symbId="eta_E">
                    <ct:VariabilityReference>
                        <ct:SymbRef symbIdRef="residual"/>
                    </ct:VariabilityReference>
                    <NormalDistribution xmlns="http://www.uncertml.org/3.0" 
                    definition="http://www.uncertml.org/3.0">
                        <mean>
                            <rVal>0</rVal>
                        </mean>
                        <stddev>
                            <var varId="a2"/>
                        </stddev>
                    </NormalDistribution>
                </RandomVariable>
                <General symbId="Effect">
                    <ct:Assign>
                        <math:Equation>
                            <math:Binop op="plus">
                                <ct:SymbRef symbIdRef="E"/>
                                <ct:SymbRef symbIdRef="eta_E"/>
                            </math:Binop>
                        </math:Equation>
                    </ct:Assign>
                </General>
            </ContinuousData>
        </ObservationModel>
        
        <!-- Discrete model for Level -->
        <ObservationModel blkId="om3">
            <Discrete>
                <CategoricalData ordered="no">
                    <ListOfCategories>
                        <Category symbId="cat1"/>
                        <Category symbId="cat2"/>
                        <Category symbId="cat3"/>
                    </ListOfCategories>
                    
                    <CategoryVariable symbId="Level"/>
                    
                    <!-- logit(P(Level<=1)) = lp1-->
                    <ProbabilityAssignment>
                        <Probability linkFunction="logit">
                            <math:LogicBinop op="leq">
                                <ct:SymbRef symbIdRef="Level"/>
                                <ct:SymbRef symbIdRef="cat1"/>
                            </math:LogicBinop>
                        </Probability>
                        <ct:Assign>
                            <ct:SymbRef blkIdRef="sm1" symbIdRef="lp1"/>
                        </ct:Assign>
                    </ProbabilityAssignment>
                    
                    <!-- logit(P(Level<=2)) = lp2-->
                    <ProbabilityAssignment>
                        <Probability linkFunction="logit">
                            <math:LogicBinop op="leq">
                                <ct:SymbRef symbIdRef="Level"/>
                                <ct:SymbRef symbIdRef="cat2"/>
                            </math:LogicBinop>
                        </Probability>
                        <ct:Assign>
                            <ct:SymbRef blkIdRef="sm1" symbIdRef="lp2"/>
                        </ct:Assign>
                    </ProbabilityAssignment>
                </CategoricalData>
            </Discrete>
        </ObservationModel>
\end{lstlisting}



%%%%%%%%%%%%%%%%%%%%%%%%%%%%%%%%%%%%%%%%%%%%%%%%%%%%%%%%%%%%%%%%
\subsection{Latent variable model}
\label{subsec:latentVariableModel}

\paragraph{Model description}

\begin{itemize}
\item
Type of observed variable -- continuous
\item
Continuous variables: $ACRL$
\item
Structural model
\begin{align}
& \frac{d ACRL}{dt} = k_{in} \times IH - k_{out} \times ACRL(t) \nonumber \\
& IH = 1 - plb - drug \nonumber \\
& plb = plb_{max} (1 - \exp(-k_{plb} \times t)) \nonumber \\
& drug = \frac{Emax \times Cp}{EC50 + Cp} \nonumber 
\end{align}
\end{itemize}

\paragraph{Observation model}

\begin{itemize}
\item
Type of observed variable -- discrete/categorical
\item
Category variable: $r$
\item
Set of categories: $\{ACR20,ACR50,ACR70\}$
\item
Exact logits
\begin{align}
& \text{logit}(P(r = ACRr)) = \text{logit}(1-r/100) + \text{logit}(1-ACRL) + \eta \nonumber \\
& \text{with} \quad \text{logit}(1 - ACRL) \quad  \text{as the} \; \textbf{latent} \; \text{variable} \nonumber 
\end{align}
\end{itemize}


\subsubsection{MLXTRAN code:}
\myStartLine

\lstset{language=MLXTRANcode}
\begin{lstlisting}
...
\end{lstlisting}
\myEndLine

\subsubsection{NMTRAN code:}
\myStartLine

\lstset{language=NONMEMdataSet}
\begin{lstlisting}
...
\end{lstlisting}
\myEndLine

\subsubsection{PharmML code:}
The structural model is shown for clarity.

\lstset{language=XML}
\begin{lstlisting}
        <StructuralModel blkId="sm1">
            <!-- follwing equations skipped here        
                dACRL/dt =  kin*IH-kout*ACRL(t)
                IH = 1-plb-drug;
                plb = plb_{max}*(1-exp(-k_{plb}*t))
                drug = (Emax*Cp)/(EC50+Cp)-->
        </StructuralModel>
        
        <!-- NOMINAL -->
        <ObservationModel blkId="om1">
            <Discrete>
                <CategoricalData ordered="yes">
                    <RandomVariable symbId="eta">
                        <ct:VariabilityReference>
                            <ct:SymbRef symbIdRef="subject"/>
                        </ct:VariabilityReference>
                        <un:NormalDistribution definition="http://www.uncertml.org/3.0">
                            <un:mean><un:rVal>0</un:rVal></un:mean>
                            <un:stddev><un:var varId="omega"/></un:stddev>
                        </un:NormalDistribution>
                    </RandomVariable>
                    <ListOfCategories> 
                        <Category symbId="cat1"/>        <!-- ACR20 -->
                        <Category symbId="cat2"/>        <!-- ACR50 -->
                        <Category symbId="cat3"/>        <!-- ACR70 -->
                    </ListOfCategories>
                    
                    <CategoryVariable symbId="r"/>
                    
                    <!-- logit(P(r=cat1)) = logit(1-r/100) + logit(1-ACRL) + eta -->
                    <ProbabilityAssignment>
                        <Probability linkFunction="identity">
                            <math:LogicBinop op="eq">
                                <ct:SymbRef symbIdRef="r"/>
                                <ct:SymbRef symbIdRef="cat1"/>
                            </math:LogicBinop>
                        </Probability>
                        <ct:Assign>
                            <math:Equation>
                                <math:Binop op="plus">
                                    <math:Uniop op="logit">
                                        <math:Binop op="minus">
                                            <ct:Real>1</ct:Real>
                                            <math:Binop op="divide">
                                                <ct:SymbRef symbIdRef="r"/>
                                                <ct:Real>100</ct:Real>
                                            </math:Binop>
                                        </math:Binop>
                                    </math:Uniop>
                                    <math:Binop op="plus">
                                        <math:Uniop op="logit">
                                            <math:Binop op="minus">
                                                <ct:Real>1</ct:Real>
                                                <ct:SymbRef symbIdRef="ACRL"/>
                                            </math:Binop>
                                        </math:Uniop>
                                        <ct:SymbRef symbIdRef="eta"/>
                                    </math:Binop>
                                </math:Binop>
                            </math:Equation>
                        </ct:Assign>
                    </ProbabilityAssignment>
                    
                    <!-- assignments for two other categories skipped -->
                    <!-- logit(P(y=cat2)) = logit(1-r/100) + logit(1-ACRL) + eta -->
                    <!-- logit(P(y=cat3)) = logit(1-r/100) + logit(1-ACRL) + eta -->
                    
                </CategoricalData>
            </Discrete>
\end{lstlisting}
            
            

%% TEMPLATE for CATEGORICAL MODEL DESCRIPTION
%\begin{itemize}
%\item
%Type of observed variable -- discrete/categorical
%\item
%Set of categories, e.g. $Y = \{0,1\}$ or $Y = \{1,2,3\}$
%\item
%Dependence, i.e. Markovian
%%\item
%%Markovian dependence - yes/no
%%\begin{itemize}
%%\item
%%Constant or variable order
%%\item
%%Order of Markovian 
%%\end{itemize}
%%\item
%%Set of $k$ values taken by the observed variable, e.g. $Y = \{0,1\}$ or $Y = \{1,2,3\}$ etc.
%%\item
%%Parameters to be estimated, e.g. \{a1, a2, a3\} or \{$\theta_1$, $\theta_2$\}
%\item
%Probability distributions for each observation
%\begin{itemize}
%\item
%Categorical data models
%\begin{itemize}
%\item
%Binomial distribution, e.g. P(Y=1) = p (for $y \in \{0,1\}$) 
%\item
%Explicit probabilities for the k-1 categories, here for $y \in \{1,2,3\}$, 
%\begin{eqnarray}
%&&P(y=1)=a1/(a1+a2+a3) \nonumber \\
%&&P(y=2)=a2/(a1+a2+a3)  \nonumber \\
%&&P(y=3)=1-P(y=1)-P(y=2) \nonumber
%\end{eqnarray}
%\end{itemize}
%\item
%Ordered categorical data models for e.g. $y \in \{1,2,3\}$
%\begin{itemize}
%\item
%Cumulative probabilities, e.g. 
%\begin{eqnarray}
%&&P(y\leq1)=a1/(a1+a2+a3) \nonumber \\
%&&P(y\leq2)=(a1+a2)/(a1+a2+a3) \nonumber \\
%&&P(y\leq3)=1-P(y<=2) \nonumber
%\end{eqnarray}
%%\item
%%Tail probabilities, e.g. specifying
%%\begin{eqnarray}
%%&& P(y>1)=(a2+a3)/(a1+a2+a3) \nonumber \\
%%&& P(y>2)=a3/(a1+a2+a3)  \nonumber \\ 
%%&& P(y>3)=1-P(y>2) \nonumber
%%\end{eqnarray}
%\item
%Logit probabilities, e.g. 
%\begin{eqnarray}
%&& \logit(P(y\leq1))= \theta_1  \nonumber \\
%&& \logit(P(y\leq2))= \theta_1+\theta_2  \nonumber \\
%&& \logit(P(y\leq3))=1  \nonumber
%\end{eqnarray}
%\end{itemize}
%\end{itemize}
%\end{itemize}

%%% TEMPLATE for CATEGORICAL MODEL DESCRIPTION
%\begin{itemize}
%\item
%Type of observed variable -- discrete/categorical
%\item
%Set of categories, e.g. $y = \{0,1\}$ or $y = \{1,2,3\}$
%\item
%Dependence, i.e. Markovian
%\item
%Set of $k$ values taken by the observed variable, e.g. $y = \{0,1\}$ or $y = \{1,2,3\}$ etc.
%\item
%Probability distributions for each observation
%\begin{itemize}
%\item
%Categorical data models
%\begin{itemize}
%\item
%Binomial distribution, e.g. P(y=1) = p (for $y \in \{0,1\}$) 
%\item
%Explicit probabilities for the k-1 categories, here for $y \in \{1,2,3\}$, 
%\begin{eqnarray}
%&&P(y=1)=a1/(a1+a2+a3) \nonumber \\
%&&P(y=2)=a2/(a1+a2+a3)  \nonumber \\
%&&P(y=3)=1-P(y=1)-P(y=2) \nonumber
%\end{eqnarray}
%\end{itemize}
%\item
%Ordered categorical data models for e.g. $y \in \{1,2,3\}$
%\begin{itemize}
%\item
%Cumulative probabilities, e.g. 
%\begin{eqnarray}
%&&P(y\leq1)=a1/(a1+a2+a3) \nonumber \\
%&&P(y\leq2)=(a1+a2)/(a1+a2+a3) \nonumber \\
%&&P(y\leq3)=1-P(y<=2) \nonumber
%\end{eqnarray}
%\item
%Logit probabilities, e.g. 
%\begin{eqnarray}
%&& \logit(P(y\leq1))= \theta_1  \nonumber \\
%&& \logit(P(y\leq2))= \theta_1+\theta_2  \nonumber \\
%&& \logit(P(y\leq3))=1  \nonumber
%\end{eqnarray}
%\end{itemize}
%\end{itemize}
%\end{itemize}


