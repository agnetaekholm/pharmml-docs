\section{PharmML templates}
\label{sec:codeTemplates}

Model templates are intended to provide a quick overview of the model types covered by this 
\pharmml version. They are simpler to read then the full examples in next sections because
here only essential elements of the structure are shown without the often lengthily 
mathematical formulas. These templates can be used as initial help for model encoding
or just to understand the overall structure.

\subsection{Count data -- basic example}
Simplified PharmML code for a basic Poisson count data models
\lstset{language=XML}
\begin{lstlisting}
        <ObservationModel blkId="om1">
            <Discrete>
                <CountData>
                    <CountVariable symbId="Y"/>

                    <IntensityParameter symbId="Lambda">
                        <!-- e.g. assignment for Lambda -->
                    </IntensityParameter>
                    
                    <!-- Using UncertML -->
                    <PMF linkFunction="log">
                        <math:LogicBinop op="eq">
                            <ct:SymbRef symbIdRef="Y"/>
                            <ct:SymbRef symbIdRef="k"/>
                        </math:LogicBinop>
                        <PoissonDistribution xmlns="http://www.uncertml.org/3.0">
                            <rate>
                                <var varId="Lambda"/>
                            </rate>
                        </PoissonDistribution>
                    </PMF>
                    
                    <!-- ALTERNATIVELY explicit implementation -->
                    <!-- log(P(Y=k)) = -Lambda+k*log(Lambda)-log(k!)-->
                    <PMF linkFunction="log">
                        <math:LogicBinop op="eq">
                            <ct:SymbRef symbIdRef="Y"/>
                            <ct:SymbRef symbIdRef="k"/>
                        </math:LogicBinop>
                        <ct:Assign>
                            <!-- -Lambda+k*log(Lambda)-log(k!) -->
                        </ct:Assign>
                    </PMF>
                </CountData>
            </Discrete>
        </ObservationModel>
\end{lstlisting}


\subsection{Count data -- over-dispersed data models}
Simplified PharmML code for count data models with over-dispersed data
\lstset{language=XML}
\begin{lstlisting}
        <ObservationModel blkId="om2">
            <Discrete>
                <CountData>
                    <CountVariable symbId="Y"/>
                    <PreviousCountVariable symbId="Yp"/>

                    <Dependance type="continuousMarkov">
                        <!-- ... or discreteMarkov -->
                    </Dependance>
                    
                    <IntensityParameter symbId="Lambda">
                        <!-- e.g. assignment for Lambda -->
                    </IntensityParameter>
                    
                    <DispersionParameter symbId="Delta">
                        <!-- e.g. assignment for Delta -->
                    </DispersionParameter>

                    <OverDispersionParameter symbId="Tau">
                        <!-- e.g. assignment for Tau -->
                    </OverDispersionParameter>
                    
                    <ZeroProbabilityParameter symbId="P0">
                        <!-- e.g. assignment for  -->
                    </ZeroProbabilityParameter>
                    
                    <MixtureProbabilityParameter symbId="Pi">
                        <!-- e.g. assignment for Pi -->
                    </MixtureProbabilityParameter>
                    
                    <!-- Using UncertML -->
                    <!-- log(P(Y=k)) = -Lambda+k*log(Lambda)-log(k!) -->
                    <PMF linkFunction="log">
                        <math:LogicBinop op="eq">
                            <ct:SymbRef symbIdRef="Y"/>
                            <ct:SymbRef symbIdRef="k"/>
                        </math:LogicBinop>
                        <PoissonDistribution xmlns="http://www.uncertml.org/3.0">
                            <rate>
                                <var varId="Lambda"/>
                            </rate>
                        </PoissonDistribution>
                    </PMF>
                    
                    <!-- ALTERNATIVELY explicit implementation -->
                    <!-- log(P(Y=k)) = -Lambda+k*log(Lambda)-log(k!) -->
                    <PMF linkFunction="log">
                        <math:LogicBinop op="eq">
                            <ct:SymbRef symbIdRef="Y"/>
                            <ct:SymbRef symbIdRef="k"/>
                        </math:LogicBinop>
                        <ct:Assign>
                            <!-- -Lambda+k*log(Lambda)-log(k!) -->
                        </ct:Assign>
                    </PMF>
                </CountData>
            </Discrete>
        </ObservationModel>
\end{lstlisting}
Note, \marginpar{\HandCuffLeft} that the use of named distribution parameters elements as in the template, 
such as \xelem{IntensityParameter}, \xelem{DispersionParameter} and others, listed in section 
\ref{subsec:mmCountData}, is optional. 
They provide the advantage that the translation engine to any other target tool will immediately 
recognise their role in the model which enhances its translation and the handling of the models in the 
target tool.

\subsection{Nominal categorical data}
Simplified PharmML code for nominal categorical models
\lstset{language=XML}
\begin{lstlisting}
        <ObservationModel blkId="om1">
            <Discrete>
                <CategoricalData ordered="yes">
                    <SimpleParameter symbId="p"/>
                    
                    <ListOfCategories> 
                        <Category symbId="cat0"/>
                        ...
                    </ListOfCategories>
                    
                    <CategoryVariable symbId="y"/>
                    
                    <!-- P(y = 1) = p -->
                    <ProbabilityAssignment>
                        <Probability linkFunction="identity">
                            <math:LogicBinop op="eq">
                                <ct:SymbRef symbIdRef="y"/>
                                <ct:SymbRef symbIdRef="cat1"/>
                            </math:LogicBinop>
                        </Probability>
                        <ct:Assign>
                            <ct:SymbRef symbIdRef="p"/>
                        </ct:Assign>
                    </ProbabilityAssignment>
                    
                    <!-- ALTERNATIVE -->
                    <!-- Using UncertML -->
                    <PMF linkFunction="identity">
                        <un:BernoulliDistribution definition="http://www.uncertml.org/3.0">
                            <un:categoryProb definition="http://www.uncertml.org/3.0">
                                <un:name>cat0</un:name>
                                <un:probability>
                                    <un:var varId="p"/>
                                </un:probability>
                            </un:categoryProb>
                        </un:BernoulliDistribution>
                    </PMF>  
                </CategoricalData>
            </Discrete>
        </ObservationModel>
\end{lstlisting}
As before, the alternative way \marginpar{\HandCuffLeft} to encode categorical models is to use 
the assignment based notation, as common 
in NONMEM. The user can simply define standard PharmML elements, such as \xelem{SimpleParameter} or 
\xelem{Variable} instead. The disadvantage is the lower information content of such code and its limited
usability in other tools, see detailed discussion in Section \ref{sec:twoModes}.

\subsection{Ordered categorical data}
Simplified PharmML code for ordered categorical models with cumulative probabilities
\lstset{language=XML}
\begin{lstlisting}
        <ObservationModel blkId="om2">
            <Discrete>
                <CategoricalData ordered="yes">
                    <ListOfCategories> 
                        <Category symbId="cat1"/>
                        <!-- omitted other assignments -->
                    </ListOfCategories>
                    
                    <CategoryVariable symbId="y"/>
                    
                    <!-- P(y <= 1) = a1/(a1+a2+a3) --> 
                    <ProbabilityAssignment>
                        <Probability>
                            <math:LogicBinop op="leq">
                                <ct:SymbRef symbIdRef="y"/>
                                <ct:SymbRef symbIdRef="cat1"/>
                            </math:LogicBinop>
                        </Probability>
                        <ct:Assign>
                            <math:Equation>
                                <math:Binop op="divide">
                                    <ct:SymbRef symbIdRef="a1"/>
                                    <math:Binop op="plus">
                                        <ct:SymbRef symbIdRef="a1"/>
                                        <math:Binop op="plus">
                                            <ct:SymbRef symbIdRef="a2"/>
                                            <ct:SymbRef symbIdRef="a3"/>
                                        </math:Binop>
                                    </math:Binop>
                                </math:Binop>
                            </math:Equation>
                        </ct:Assign>
                    </ProbabilityAssignment>
                </CategoricalData>
            </Discrete>
        </ObservationModel>
\end{lstlisting}

\subsection{Ordered categorical data with Markov dependency ($1^{st}$ and $2^{nd}$ order)}
Simplified PharmML code for ordered categorical model with cumulative probabilities 
and discrete time Markov dependency for 1st and 2nd order Markov models.
\lstset{language=XML}
\begin{lstlisting}
        <ObservationModel blkId="om3">
            <Discrete>
                <CategoricalData ordered="yes">
                    <SimpleParameter symbId="p01"/>
                    <SimpleParameter symbId="a112"/>
                        <!-- omitted parameters -->
                    
                    <ListOfCategories>
                        <Category symbId="cat1"/>
                    </ListOfCategories>
                    
                    <CategoryVariable symbId="y"/>
                    <InitialStateVariable symbId="yinit"/>
                    <PreviousStateVariable symbId="yp1"/>
                    <PreviousStateVariable symbId="yp2"/>
                    
                    <Dependance type="discreteMarkov"/>
                    
                    <!-- Initial state probability (optional) -->
                    <!-- P(y = 1) = a1 -->
                    <ProbabilityAssignment>
                        <Probability>
                            <!-- yinit = 1 -->
                        </Probability>
                        <ct:Assign>
                            <ct:SymbRef symbIdRef="a1"/>
                        </ct:Assign>
                    </ProbabilityAssignment>
                    
                    <!-- 1st order -->
                    <!-- P(y<=1|yp1=0)=p01 -->
                    <ProbabilityAssignment>
                        <Probability>
                            <CurrentState>
                                <!-- y<=1 -->
                            </CurrentState>
                            <PreviousState>
                                <!-- yp1=0 -->
                            </PreviousState>
                        </Probability>
                        <ct:Assign>
                            <ct:SymbRef symbIdRef="p01"/>
                        </ct:Assign>
                    </ProbabilityAssignment>
                    
                    <!-- 2nd order -->
                    <!-- logit(P(y <= 1 | yp1 = 1, yp2 = 2)) = a112 -->
                    <ProbabilityAssignment>
                        <Probability linkFunction="logit">
                            <CurrentState>
                                <!-- y<=1 -->
                            </CurrentState>
                            <PreviousState MarkovOrder="1">
                                <!-- yp1=1 -->
                            </PreviousState>
                            <PreviousState MarkovOrder="2">
                                <!-- yp2=2 -->
                            </PreviousState>
                            <Condition>
                                <!-- other condition -->
                            </Condition>
                        </Probability>
                        <ct:Assign>
                            <ct:SymbRef symbIdRef="a112"/>
                        </ct:Assign>
                    </ProbabilityAssignment>
                </CategoricalData>
            </Discrete>
        </ObservationModel>
\end{lstlisting}

\subsection{Ordered categorical data with continuous time Markov dependency}
Simplified PharmML code for ordered categorical model with cumulative probabilities 
and continuous Markov dependency.
\lstset{language=XML}
\begin{lstlisting}
        <ObservationModel blkId="om4">
            <Discrete>
                <CategoricalData ordered="yes">
                    <SimpleParameter symbId="p01"/>
                    
                    <ListOfCategories>
                        <Category symbId="cat0"/>
                    </ListOfCategories>
                    
                    <CategoryVariable symbId="y"/>
                    <PreviousStateVariable symbId="yp1"/>
                    
                    <Dependance type="continuousMarkov"/>
                    
                    <!-- rho(y2,y1) = exp(a+b*t) -->
                    <ProbabilityAssignment>
                        <TransitionRate>
                            <CurrentState>
                                <!-- y=1 -->
                            </CurrentState>
                            <PreviousState>
                                <!-- yp1=1 -->
                            </PreviousState>
                        </TransitionRate>
                        <ct:Assign>
                            <!-- exp(a+b*t) -->
                        </ct:Assign>
                    </ProbabilityAssignment>
                </CategoricalData>
            </Discrete>
        </ObservationModel>
\end{lstlisting}

\subsection{Time-to-event data}
Simplified PharmML code template for TTE data models
\lstset{language=XML}
\begin{lstlisting}
        <ObservationModel blkId="om1">
            <Discrete>
                <TimeToEventData>
                    <EventVariable symbId="y"/>
                    
                    <HazardFunction symbId="H">
                        <!-- h0*exp(-beta*C) -->
                    </HazardFunction>
                    
                    <SurvivalFunction symbId="S">
                        <!-- exp(-lambda*t) -->
                    </SurvivalFunction>
                    
                    <Censoring censoringType="intervalCensored">    
                        <IntervalLength>
                            <!-- e.g. <ct:Real>10</ct:Real> -->
                        </IntervalLength>
                        
                        <RightCensoringTime>
                            <!-- e.g. <ct:Real>200</ct:Real> -->
                        </RightCensoringTime>
                    </Censoring>
                    
                    <MaximumNumberEvents>
                        <!-- e.g. <ct:Real>5</ct:Real> -->
                    </MaximumNumberEvents>
                </TimeToEventData>
            </Discrete>
        </ObservationModel>
\end{lstlisting}


\section{Two modes of operation {\color{red} \scshape{*}}}
\label{sec:twoModes}
As indicated before, there are many ways to implement discrete data models. We would like to 
highlight certain issues related to this fact when dealing with the categorical data models. More precisely, 
two types of encoding of the left-hand side expressions with transformations such as log or logits, 
probabilities and other terms are of special interest for us because they are used by the main target 
tools, NONMEM and Monolix. The former one uses notation based on defining variables expressing 
certain probabilities or transformed entities, while the latter is closer to the mathematical notation 
as used in math/stat textbooks. 

More specifically, NONMEM is not able to handle explicit transformation-based expressions 
so they have to be first recalculated as, in this case, cumulative probabilities and subsequently 
exact probabilities. At each step they have to be assigned to according variables. 
Table \ref{tab:twoModes} shows the code for a cumulative logit probabilities 
model as implemented in NMTRAN and MLXTRAN, illustrating these two modes.

\begin{table}[ht!]
\setlength{\tabcolsep}{15pt}
\begin{center}
\begin{tabular}{ll}
  \hline \hline
NMTRAN -- encoding of categorical models  	& MLXTRAN -- encoding of categorical models \\[-.25ex]
is based on subsequent assignments   		& is based on standard mathematical notation \\[-.25ex]
									& (alternatively, the assignment based option \\[-.25ex]
									&  is also available in Monolix) \\
  \hline
\lstset{language=NONMEMdataSet}
\begin{lstlisting}
# A0 ... A2 cumul. logits
A0 = EDRUG + B0
A1 = EDRUG + B0 + B1
A2 = EDRUG + B0 + B1+ B2
 
# CP0 ... CP2 cumul. prob.
CP0 = 1/(1+exp(-A0))
CP1 = 1/(1+exp(-A1))
CP2 = 1/(1+exp(-A2))
 
# P0 ... P3 exact probabs
P0 = CP0
P1 = CP1 - CP0
P2 = CP2 - CP1
P3 = 1 - CP2
\end{lstlisting}
&
\lstset{language=MLXTRANcode}
\begin{lstlisting}
y = {
  type = categorical
  categories = {0, 1, 2, 3}
  logit(P(y<=0)) = EDRUG + B0
  logit(P(y<=1)) = EDRUG + B0 + B1
  logit(P(y<=2)) = EDRUG + B0 + B1+ B2}
}
\end{lstlisting} 

\\
\hline
\lstset{language=MLXTRANcode}
\begin{lstlisting}
Common for both, NMTRAN and MLXTRAN models
# PK input
     EDRUG=POP_BETA*CP

# Parameters
    B0 with IIV
    B1 - fixed
    B2 - fixed
\end{lstlisting}
\\
  \hline
\end{tabular}
\caption[Comparison of differences in encoding of discrete data models.]{%
      Comparison of differences in encoding of discrete data models\tablefootnote{The examples have 
      been slightly modified compared to the previous version, without changing the conclusions resulting 
      from their analysis as described in this section.}\label{tab:}.}
\label{tab:twoModes}
\end{center}
\end{table}

%ADD FIGURE HERE
%\begin{figure}[htbp]
%\centering
%\caption{Translation of these two representations.}
%\label{fig:translatingRepresentations}
%\end{figure}

While both notations result in mathematically correct formulation of a model, there are clear 
advantages for the users and tools of the concise MLXTRAN-like syntax
\begin{itemize}
\item
The declarative format based code is easier to parse and translate to target tools because 
of its very explicit structure and doesn't require annotation in contrast to the alternative option.
\item
It is easier to understand for everyone familiar with the underlying mathematics and newcomers 
to the field of pharmacometrics not aware with the NMTRAN language.
\item
While the translation from the right-hand side to the left-hand side (Table \ref{tab:twoModes})
is straightforward, the revers direction is virtually impossible to be performed automatically.
\end{itemize}
Nevertheless, the proposed PharmML structure supports both encoding styles.  \marginpar{\HandCuffLeft} 
This means more complex structure but at the same time it makes sure that these both options 
are applicable. 

\section{Categorical data mapping}
\label{sec:catDataMapping}
Categorical models define categories using user-defined names, such as \xatt{cat1}, \xatt{cat2} 
etc., without any mapping to the dataset, in the \xelem{ObservationModel}. This allows to decouple 
the model definition from the the external dataset and increase its usability.
\lstset{language=XML}
\begin{lstlisting}
        <CategoricalData ordered="yes">
            <ListOfCategories> 
                <Category symbId="cat1"/>
                ...
            </ListOfCategories>
\end{lstlisting}
The actual mapping to the dataset is done in the \xelem{ModellinSteps} section as the the following
code snippet shows
\lstset{language=XML}
\begin{lstlisting}
            <mstep:ColumnMapping>
                <ds:ColumnRef columnIdRef="DV"/>
                <ct:SymbRef blkIdRef="om1" symbIdRef="y"/>
                <ds:CategoryMapping>
                    <ds:Map dataSymbol="0" modelSymbol="cat1"/>
                    <ds:Map dataSymbol="1" modelSymbol="cat2"/>
                    <!-- omitted other categories -->
                </ds:CategoryMapping>
            </mstep:ColumnMapping>
\end{lstlisting}
This allows keeping one model description with the possibility to use datasets
with different categories names. See also section \ref{subsec:catDataCovariatesMapping}
for more detailed description. Note, the this concept applies equally to the mapping of categorical 
covariates.

%ADD FUGURE HERE???
%\begin{figure}[htbp]
%\centering
%\caption{Mapping of categorical data.}
%\label{fig:mappingCategoricalData}
%\end{figure}


%REFERENCES:
%- publications
%e.g. \cite{Plan:2009fk}



%\begin{sidewaystable}[h]
%\caption{Performance After Post Filtering}  % title name of the table
%\centering  % centering table
%\begin{tabular}{l c c rrrrrrr}  % creating 10 columns
%\hline\hline                       % inserting double-line
% Audio &Audibility & Decision &\multicolumn{7}{c}{Sum of Extracted Bits} 
%\\ [0.5ex]   
%\hline              % inserts single-line
%% Entering 1st row
% & &soft &1 & $-1$ & 1 & 1 & $-1$ & $-1$ & 1  \\[-1ex]
%\raisebox{1.5ex}{Police} & \raisebox{1.5ex}{5}&hard
%&  2 & $-4$ & 4 & 4 & $-2$ & $-4$ & 4 \\[1ex]
%% Entering 2nd row
%& &soft & 1 & $-1$ & 1 & 1 & $-1$ & $-1$ & 1  \\[-1ex]
%\raisebox{1.5ex}{Beethoven} & \raisebox{1.5ex}{5}& hard
%&8 & $-8$ & 2 & 8 & $-8$ & $-8$ & 6 \\[1ex]
%% Entering 3rd row
%& &soft & 1 & $-1$ & 1 & 1 & $-1$ & $-1$ & 1  \\[-1ex]
%\raisebox{1.5ex}{Metallica} & \raisebox{1.5ex}{5}& hard
%&4 & $-8$ & 8 & 4 & $-8$ & $-8$ & 8  \\[1ex]
%% [1ex] adds vertical space
%\hline                          % inserts single-line
%\end{tabular}
%\label{tab:LPer}
%\end{sidewaystable}
%
