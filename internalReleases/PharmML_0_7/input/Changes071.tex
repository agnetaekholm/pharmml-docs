
\chapter{Changes in 0.7.1}
\label{ch:071changes}
The following update version complements the release of PharmML 0.7 by
few extensions or changes which will make the format more consistent 
and flexible. 

\section{Separate schema for ProbOnto}
Chapter \ref{ch:ProbOnto} describes in details ProbOnto and its features 
which remained unchanged. What changed is the embedding of ProbOnto
in PharmML. More specifically we have 
\begin{itemize}
\item 
created a separate schema for ProbOnto -- with no impact on the way
the user will work with it. The schema borrows, similarly to SO schema, concepts
from PharmML but can be otherwise developed and exteded independently.
\end{itemize}
The change means that ProbOnto, with code names hard coded in previous 
version directly in PharmML schema, is now easier to extend without the need to 
release new version of PharmML every time new distribution is added 
to ProbOnto. 

Other changes are, see appendix \ref{ch:probontoAppendix},
\begin{itemize}
\item 
added new distributions or alternative parameterisations to the existing ones -- now 
totalling to about 64 parametric distributions. This time we have considered number of
distributions used in Matlab, available in the Statistics Toolbox$^{\text{TM}}$.
\item 
fixed bugs and typos in the knowledge base
\item 
provided additional R-code which is now available for PDF/PMF and CDF for every 
univariate distribution, where available.
\item 
added according distribution plots.
\end{itemize}


\section{Removing \xelem{Equation} element}

The \xelem{Equation} element, although present since the very beginning, 
has been removed due to its redundancy (used in most cases as child 
element of \xelem{Assign}) and semantic inconsistency. Its name suggests 
that it encodes an equation but it was defined with encoding expressions in mind, 
i.e. syntactic unit of PharmML model the denotes a value. This change, 
effecting virtually every model, will moreover make reading and writing of 
models easier.

For example the following equation $X = A + B$ encoded so far
as 
\lstset{language=XML}
\begin{lstlisting}
<PopulationParameter symbId="X">
    <ct:Assign>
        <math:Equation>
            <math:Binop op="plus">
                <ct:SymbRef symbIdRef="A"/>
                <ct:SymbRef symbIdRef="B"/>
            </math:Binop>
        </math:Equation>
    </ct:Assign>
</PopulationParameter>
\end{lstlisting}
will become a bit shorter and reads now
\lstset{language=XML}
\begin{lstlisting}
<PopulationParameter symbId="X">
    <ct:Assign>
        <math:Binop op="plus">
            <ct:SymbRef symbIdRef="A"/>
            <ct:SymbRef symbIdRef="B"/>
        </math:Binop>
    </ct:Assign>
</PopulationParameter>
\end{lstlisting}
In case when \xelem{Equation} element was used without the parent \xelem{Assing} element, such as 
in function definition
\lstset{language=XML}
\begin{lstlisting}
    <ct:FunctionDefinition symbolType="real" symbId="proportional">
        <ct:FunctionArgument symbolType="real" symbId="b"/>
        <ct:FunctionArgument symbolType="real" symbId="f"/>
        <ct:Definition>
            <math:Equation>
                <math:Binop op="times">
                    <ct:SymbRef symbIdRef="b"/>
                    <ct:SymbRef symbIdRef="f"/>
                </math:Binop>
            </math:Equation>
        </ct:Definition>
    </ct:FunctionDefinition>
\end{lstlisting}
it is replaced by the latter and reads
\lstset{language=XML}
\begin{lstlisting}
    <ct:FunctionDefinition symbolType="real" symbId="proportional">
        <ct:FunctionArgument symbolType="real" symbId="b"/>
        <ct:FunctionArgument symbolType="real" symbId="f"/>
        <ct:Definition>
            <ct:Assign>
                <math:Binop op="times">
                    <ct:SymbRef symbIdRef="b"/>
                    <ct:SymbRef symbIdRef="f"/>
                </math:Binop>
            </ct:Assign>
        </ct:Definition>
    </ct:FunctionDefinition>
\end{lstlisting}


\section{Other changes}

\subsection{\xatt{columnType} in definition of the dataset optional}
If mapping of a particular column is defined then the \xatt{columnType} attribute is 
not required, and \emph{vice versa}, unless required by a target tool. This helps avoiding potential 
redundancies or inconsistencies which we stumbled upon when translating 
models from MDL to PharmML. 

\subsection{Schema refactoring}
A number of changes on the schema level has been made, such as 
removing/replacing redundant types. These changes will have usually non or a minimal 
impact on the way the users interact with PharmML coded models. 

\subsection{Extending \xelem{Sequence} notation}
Without adding a new element to the sequence structure, but with changes in its
underlying type, a new option on the top of the existing two is available for the encoding of 
numeric sequences, here in Matlab speak
\begin{description}
\item[option 1]
Begin:StepSize:End or
\item[option 2] 
Begin:StepSize:StepNumber or
\item[option 3]
{\color{red} \scshape{new}} Begin:StepNumber:End
\end{description}
Note that the \xelem{Repetitions} element has been renamed to \xelem{StepNumber}
for consistency with its use and interpretation.
With this change the original version of the Simulx model, see section \ref{sec:noArms}, 
is encodable directly without the need of recalculating the given values
\lstset{language=MLX}
\begin{lstlisting}
	Cc <- list(name='Cc', time=seq(-5, 100, length=500))
\end{lstlisting}
It can be encoded as the following code snippet shows
\lstset{language=XML}
\begin{lstlisting}
<TrialDesign>
    <Interventions>
        <Administration oid="adm1">
            <Bolus>
                <!-- details ommited here -->
                <DosingTimes>
                    <ct:Assign>
                        <ct:Sequence>
                            <ct:Begin><ct:Real>-5</ct:Real></ct:Begin>
                            <ct:StepNumber><ct:Int>500</ct:Int></ct:StepNumber>
                            <ct:End><ct:Real>100</ct:Real></ct:End>
                        </ct:Sequence>
                    </ct:Assign>
                </DosingTimes>
            </Bolus>
        </Administration>
\end{lstlisting}














