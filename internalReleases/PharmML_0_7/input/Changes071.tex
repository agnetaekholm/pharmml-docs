
\chapter{Changes in 0.7.1}


\section{Separate schema for ProbOnto}

\section{Removing \xelem{Equation} element}

Removing

\lstset{language=XML}
\begin{lstlisting}
<PopulationParameter symbId="CL">
    <ct:Assign>
        <math:Equation>
            <math:Binop op="plus">
                <ct:SymbRef symbIdRef="CLrenal"/>
                <ct:SymbRef symbIdRef="CLother"/>
            </math:Binop>
        </math:Equation>
    </ct:Assign>
</PopulationParameter>
\end{lstlisting}

\lstset{language=XML}
\begin{lstlisting}
<PopulationParameter symbId="CL">
    <ct:Assign>
        <math:Binop op="plus">
            <ct:SymbRef symbIdRef="CLrenal"/>
            <ct:SymbRef symbIdRef="CLother"/>
        </math:Binop>
    </ct:Assign>
</PopulationParameter>
\end{lstlisting}

Instead
\lstset{language=XML}
\begin{lstlisting}
    <ct:FunctionDefinition symbolType="real" symbId="proportional">
        <ct:FunctionArgument symbolType="real" symbId="b"/>
        <ct:FunctionArgument symbolType="real" symbId="f"/>
        <ct:Definition>
            <math:Equation>
                <math:Binop op="times">
                    <ct:SymbRef symbIdRef="b"/>
                    <ct:SymbRef symbIdRef="f"/>
                </math:Binop>
            </math:Equation>
        </ct:Definition>
    </ct:FunctionDefinition>
\end{lstlisting}
now
\lstset{language=XML}
\begin{lstlisting}
    <ct:FunctionDefinition symbolType="real" symbId="proportional">
        <ct:FunctionArgument symbolType="real" symbId="b"/>
        <ct:FunctionArgument symbolType="real" symbId="f"/>
        <ct:Definition>
            <ct:Assign>
                <math:Binop op="times">
                    <ct:SymbRef symbIdRef="b"/>
                    <ct:SymbRef symbIdRef="f"/>
                </math:Binop>
            </ct:Assign>
        </ct:Definition>
    </ct:FunctionDefinition>
\end{lstlisting}


\section{Extending \xelem{Sequence} notation}
A new option has been added to the first two elements, which was
missing in 0.7, see section \ref{sec:noArms}
\begin{description}
\item[option 1]
Begin:StepSize:End or
\item[option 2] 
Begin:StepSize:StepNumber or
\item[option 3]
Begin:StepNumber:End
\end{description}

With this the original version of the Simulx model is encodable which reads
\lstset{language=MLX}
\begin{lstlisting}
	Cc <- list(name='Cc', time=seq(-5, 100, length=500))
\end{lstlisting}
can be encoded as 
\lstset{language=XML}
\begin{lstlisting}
<TrialDesign>
    <Interventions>
        <Administration oid="adm1">
            <Bolus>
                <!-- details ommited here -->
                <DosingTimes>
                    <ct:Assign>
                        <ct:Sequence>
                            <ct:Begin><ct:Real>-5</ct:Real></ct:Begin>
                            <ct:StepNumber><ct:Int>500</ct:Int></ct:StepNumber>
                            <ct:End><ct:Real>100</ct:Real></ct:End>
                        </ct:Sequence>
                    </ct:Assign>
                </DosingTimes>
            </Bolus>
        </Administration>
\end{lstlisting}


\section{\xatt{columnType} in definition of the dataset optional}
If mapping of a particular column is defined as well then this 
attribute it is not required. This helps avoiding redundancies or inconsistencies.

Consider the following case from \emph{count\_PMAKmodel.xml}:
\lstset{language=XML}
\begin{lstlisting}
<ColumnMapping>
    <ds:ColumnRef columnIdRef="TIME"/>
    <ct:SymbRef symbIdRef="t"/>
</ColumnMapping>
<ColumnMapping>
    <ds:ColumnRef columnIdRef="Y"/>
    <ct:SymbRef blkIdRef="om1" symbIdRef="y"/>
</ColumnMapping>
<ds:Definition>
    <ds:Column columnId="ID" columnType="id" valueType="string" columnNum="1"/>
    <ds:Column columnId="TIME" columnType="time" valueType="real" columnNum="2"/>
    <ds:Column columnId="Y" columnType="dv" valueType="int" columnNum="3"/>
</ds:Definition>
\end{lstlisting}















