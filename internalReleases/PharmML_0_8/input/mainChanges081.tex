
\chapter{Changes in 0.8.1}
\label{ch:081changes}


\section{Extensions in \xatt{columnType} }
Changes were required because both of demands in PharmML and SO\footnote{The schema 
of the Standard Output (SO) reuses certain PharmML constructs, such as data declaration 
support but it is otherwise independent from PharmML, \cite{SO:2016b}.}.

\begin{itemize}
\item 
new values of the \xatt{columnType} attribute value \emph{varLevel} and (only relevant for SO) \emph{variance}, \emph{stdev}, \emph{mode}, \emph{median	}
\item
using multiple values in combination e.g.
\lstset{language=XML}
\begin{lstlisting}
		columnType="covariate varLevel"
\end{lstlisting}
\end{itemize}
See next section for a proper application case.


\section{Higher level of variability}

\subsection{Implicit mapping}

Unambiguous identification of covariates mapped to level of variability 

Previously only \emph{occasion} value was available which limited the usefulness
of \xatt{columnType} specification. In the case an occasion was supposed to 
be used as covariate and additional variability level source only 

Also identifying higher levels of variability wouldn't be possible directly 
via \xatt{columnType} and only indirectly via \xelem{ColumnMapping}.

These cases are illustrated with the following example

\lstset{language=XML}
\begin{lstlisting}
        <ds:Definition>
            <ds:Column columnId="ID" columnType="id" valueType="string" columnNum="1"/>
            <ds:Column columnId="TIME" columnType="time" valueType="real" columnNum="2"/>
            <ds:Column columnId="Y" columnType="dv" valueType="real" columnNum="3"/>
            <ds:Column columnId="AMT" columnType="dose" valueType="real" columnNum="4"/>
            <ds:Column columnId="OCC" columnType="covariate" valueType="int" columnNum="5"/>
        </ds:Definition>
\end{lstlisting}
with double mapping of OCC
\begin{itemize}
\item 
covariate mapping 
\item 
variability model mapping 
\end{itemize}

\lstset{language=XML}
\begin{lstlisting}
            <ColumnMapping>
                <ds:ColumnRef columnIdRef="OCC"/>
                <ct:SymbRef blkIdRef="cm1" symbIdRef="Occasion"/>
                <ds:CategoryMapping>
                    <ds:Map modelSymbol="occ1" dataSymbol="1"/>
                    <ds:Map modelSymbol="occ2" dataSymbol="2"/>
                </ds:CategoryMapping>
            </ColumnMapping>
            <ColumnMapping>
                <ds:ColumnRef columnIdRef="OCC"/>
                <ct:SymbRef blkIdRef="vm1" symbIdRef="iov1"/>
            </ColumnMapping>
\end{lstlisting}


\subsection{Explicit mapping}

\lstset{language=XML}
\begin{lstlisting}
        <ds:Definition>
            <ds:Column columnId="ID" columnType="id" valueType="string" columnNum="1"/>
            <ds:Column columnId="TIME" columnType="time" valueType="real" columnNum="2"/>
            <ds:Column columnId="Y" columnType="dv" valueType="real" columnNum="3"/>
            <ds:Column columnId="AMT" columnType="dose" valueType="real" columnNum="4"/>
            <ds:Column columnId="OCC" columnType="covariate varLevel" valueType="int" columnNum="5"/>
        </ds:Definition>
\end{lstlisting}


\section{Optimal design extensions}
\label{sec:OEDextentions}

Optimal design and related task has been extended with few missing components.

\subsection{Parameter settings}
\xelem{ParametersToEstimate} added to optional design task description -- an element
featured in estimation task already. Its reuse here allows for specification of parameters
initial values, lower/upper bound values assignment or indication about 
parameters to be kept fixed.


\subsection{Added stage definition}

The following examples are borrowed from \cite{CommetsExamples2015}
and visualises the option available for the \xelem{StageDefinition} element
\begin{itemize}
\item 
single stage specification
\lstset{language=MLX}
\begin{lstlisting}
        DS{name=t1, element=doseTime, range=[0,6], stage = 1 }
\end{lstlisting}
with PharmML implementation
\lstset{language=XML}
\begin{lstlisting}
        <StageDefinition>
            <math:LogicBinop op="eq">
                <math:Stage/>
                <ct:Real>1</ct:Real>
            </math:LogicBinop>
        </StageDefinition>
\end{lstlisting}

\item 
set stage specification
\lstset{language=MLX}
\begin{lstlisting}
        DS{name=t1, element=doseTime, range=[0,8], stage = {1,2} }
\end{lstlisting}
\lstset{language=XML}
with PharmML implementation
\begin{lstlisting}
        <StageDefinition>
            <math:LogicBinop op="eq">
                <design:Stage/>
                <ct:Vector>
                    <ct:VectorElements>
                        <ct:Real>1</ct:Real>
                        <ct:Real>2</ct:Real>
                        <ct:Real>3</ct:Real>
                    </ct:VectorElements>
                </ct:Vector>
            </math:LogicBinop>
        </StageDefinition>
\end{lstlisting}

\item 
interval stage specification
\lstset{language=MLX}
\begin{lstlisting}
        DS{name=t1, element=doseTime, range=[0,8], stage > 1 & stage < 4 }
\end{lstlisting}
\lstset{language=XML}
with PharmML implementation
\begin{lstlisting}
        <StageDefinition>
            <math:LogicBinop op="or">
                <math:LogicBinop op="gt">
                    <Stage/>
                    <ct:Int>1</ct:Int>
                </math:LogicBinop>
                <math:LogicBinop op="lt">
                    <Stage/>
                    <ct:Int>4</ct:Int>
                </math:LogicBinop>
            </math:LogicBinop>
        </StageDefinition>
\end{lstlisting}
\end{itemize}

\section{Minor changes}

\begin{itemize}
\item 
Adding software settings element \xelem{SoftwareSettings} to common task 
types allowing for optional storage of software specific settings -- in agreement 
with recent MDL task proposal.
\item
Type defining the FIM-atrix with attribute \xatt{type} being assigned values \{B, I, P\}, e.g.
\lstset{language=XML}
\begin{lstlisting}
			<FIM type="P"/>
\end{lstlisting}
\end{itemize}


